\documentclass{book}
\usepackage[OT2]{fontenc}
\usepackage[utf8]{inputenc}
\usepackage[russian,english]{babel}
\usepackage{graphicx}
%\documentclass[a5paper][ebook,oneside,openany]{memoir}

\graphicspath{C:/Users/volodymyr.turbaevsky/Desktop/pict}
%\graphicspath{/home/tur/trader-blogger.com/trader-blogger.com/wp-content/uploads/2014/05/}

\begin{document}
Cейчас вы находитесь на главной странице, которая поможет вам разобраться с дальнейшими действами. На блоге находится несколько сотен статей, страниц и прочего контента, который я постарался как можно удобнее для вас сгруппировать. Все дороги к полезной информации ведут отсюда.

Итак, прежде всего я вам рекомендую ознакомиться со страницей «Автор и блог». Там вы найдете необходимую информацию обо мне, о причинах создания и миссии этого блога. Будем знакомы ближе!

Очень важен следующий момент: заходите на страницу «Оставить отзыв» и пишите свои комментарии в обязательном порядке. Блог – это не монолог одного человека. Есть люди, которые попали сюда случайно и больше не возвращаются. Если вы не один из них, то просто обязаны оставить свой отзыв и вот почему:

Вы показываете мне, что именно вас интересует. Поэтому я могу писать на интересующие вас темы
Создавая дискуссии и качественно дополняя существующие материалы, вы помогаете развиваться и себе, и другим трейдерам более эффективно.
Короче, думаем о ближнем своем и комментируем. Это также касается и каждого поста.

И еще одно. Страница «Контакты» созданы для вопросов типа: «Дмитрий, напиши о …» или «А можно по этой теме записать видеоролик, а то текст сложен для понимания?» и т.п. Прошу вас, не писать личные сообщения, или типа: «Что за бред ты здесь несешь», или «Да, полезный блог». Все это не несет абсолютно никакой смысловой нагрузки и не помогает в развитии ресурса. Не стесняйтесь писать свои истинные соображения по теме.

А теперь перейдем к разделам блога. Главная тема – это, конечно же, свинг трейдинг, который разбит на три рубрики (основы, стратегия и продвинутый уровень). Но, существует еще как минимум десяток других обучающих разделов, которые созданы для того, чтобы начинающий трейдер смог эффективно и глубоко разобраться в теме торговли акциями, понять ее риски и возможности. Теперь кратко по каждому разделу:

Свинг трейдинг: основы. Нельзя приступать к стратегии, пока не понимаешь ее основных рабочих винтиков. Здесь мы рассмотрим те базовые понятия, которые помогут нам осознать принципы успешного трейдинга. Особое внимание обратите на темы уровней поддержки и сопротивления и свечных графиков.
Свинг трейдинг: стратегия. Кто бы и что бы вам не говорил, но существует две основные стратегии торговли акциями: пробой уровня и откат от уровня. Я выбираю торговлю на откате и в этом разделе объясняю почему. Здесь вы получите комплексную стратегию свинг трейдинга, включая открытие и закрытие позиции, ведение сделки на случай как положительного, так и негативного исхода.
Продвинутый свинг трейдинг. Я собрал некоторые ценные рекомендации по ценовым движениям, паттернам и полезным интернет ресурсам в этом разделе. Здесь вам главное не распыляться, а взять только то, что вам действительно покажется необходимым для торговли. Основная фишка: автоматический отбор акций для стратегии свинг трейдинга.
Thinkorswim – демо-счет. Трейдер – это на 90\% практик. Вы можете каждый день сидеть возле компьютера в поисках магической формулы трейдинга, но результата это вам не принесет. Thinkorswim – одна из лучших торговых платформ в мире. В этом разделе я покажу вам, как ее получить абсолютно бесплатно, настроить и использовать. Начинайте изучать графики и совершать сделки на виртуальном счете. Если соберется достаточное количество голосов нуждающихся (пишите в отзывы или контакты), я выпущу видео-курс с подробными уроками по этой платформе.
Риск менеджмент. Без этого никуда. Это, наверное, самый большой раздел, название которого говорит само за себя.
Японские свечи. Свечной анализ мы применяем для эффективного выявления разворотных точек на тренде. Это основа и вам ее необходимо знать. Здесь я даю краткий конспект по основным свечным моделям, а также показываю, как их можно применять вместе с уровнями поддержки и сопротивления.
Торговые инструменты. Чем больший риск, тем большая возможная прибыль. Это правило хорошо известно.  Но финансовые рынки не всегда эффективны, и находятся бумаги, которые при равных рисках приносят разные доходы. В этом разделе я не просто хочу вам рассказать, что «эта бумажка – фьючерс, а эта – валютная пара». Важно, чтобы вы поняли риск, сопряженный с торговлей каждой ценной бумаги, и какие у вас есть шансы превратить его в прибыль.
Акции. Инструмент, на котором построило состояние не одно поколение инвесторов, трейдеров и бизнесменов. Разные группы биржевых игроков воспринимают акции по-разному. Как воспринимают их трейдеры? А как инвесторы? Какие есть виды акций? Все это здесь.
Анализ рынка акций. Нам доступны всевозможные техники для оценки и выбора акций для торговли. Можно использовать технический или фундаментальный анализ, торговать биржевые новости или советы финансовых аналитиков. В этом разделе я постараюсь рассказать вам об основных принципах и методах анализа фондового рынка.
Технический анализ. Я разложил данную тему на два раздела:
Технический анализ: фигуры (паттерны, модели). Я очень люблю торговать от четких уровней, которые, как правило, формируются вокруг каких-то графических моделей. Это очень важно. Иногда я даже не обращаю внимания на тип фигуры – главное, чтобы был уровень.
Технический анализ: индикаторы. У меня есть четкое правило: никаких индикаторов при просмотре и торговле акций. Ничто не должно отвлекать от ценовой динамики. Но без них никак не обойтись во время отбора акций. В общем, здесь вы найдите разные группы и типы индикаторов.
Фондовая биржа. Очень важный раздел. Не нужно его недооценивать. Вы прочтете несколько статей, чтобы понять одну главную мысль – американский рынок сегодня предлагает наилучшие условия частному трейдеру. Беря во внимание тот факт, что в эру интернета нет больших проблем открыть счет в брокера США, глупо не пользоваться лучшими возможностями.
Трейдинг. Раздел, в котором собирается общая информация по трейдингу без узкой специализации, как например, свинг трейдинг. Здесь вы можете узнать информацию о разработке торговых стратегий, стилях трейдинга (дейтрейдинг, позиционный трейдинг) и т.п.
Советы трейдеру. И так все понятно.
Фильмы и книги. Так же не должно возникнуть вопросов о содержании данного раздела.
Блог. Сюда войдут посты вне разделов, но не менее информативные.
Важно: статьи на блоге о трейдинге обновляются каждую неделю. Чтобы не пропустить их, подписывайтесь на автоматическую рассылку по почте. Никакого спама. Всем подписчикам время от времени будут приходить какие-то полезные мини-книги или видеоролики на почтовый адрес, которых не найти на страницах блога. Не пропустите!

Также мы есть в социальных сетях (иконки в самом верхнем поле каждой страницы), где вы можете с нами дружить. Так что присоединяйтесь! В принципе на этом все. Блог о трейдинге благодарит за внимание. Будьте успешными!

Свинг трейдинг — это один из видов торговых стилей, который стоит между дейтрейдингом и позиционным трейдингом. Результаты опросов, проведенных в США, показали, что все большее количество частных трейдеров переходит от внутридневной торговли, которая была очень популярна в 90-х прошлого века и начале XXI, к более долгосрочной торговли. Это дает несколько преимуществ, о которых мы будем говорить в данном обучающем разделе, главное из которых — увеличение отношения прибыли к риску. На этом сайте вы найдете полную и комплексную стратегию свинг трейдинг, с которой можно добиться хороших результатов на рынке акций, приложив должных усилий и усердия. Начинаем, конечно, с основ.

\section{Что такое свинг трейдинг и его основные принципы}

Здравствуйте, читатели блога о трейдинге. Я начал заниматься свинг трейдингом в 2012 году и продолжаю по сей день. Перешел я сюда с дейтрейдинга, которым началось мое знакомство с торговлей акциями.

В отличие от дейтрейдинга, свинг трейдинг занимает меньше вашего времени (у меня уходит около часа на день), не нужна полная занятость и полдня смотреть в монитор. При этом сохраняется значительный потенциал прибыльности для среднестатистического трейдера. Позиция в среднем держится около 4 дней, хотя бывает и 2-3 недели. Свинг трейдеры, как правило, руководствуются техническим анализом для открытия-закрытия позиции.

\subsection{Основные принципы свинг трейдинга}

Свинг трейдинг не является торговой стратегией, как некоторые считают. Это просто отдельный стиль или вид торговли ценными бумагами, так как и дейтрейдинг или позиционный трейдинг. То, как вы будете входить в позицию и выходить с нее, управлять рисками и торговым капиталом, и будет составлять вашу стратегию свинг трейдинга.

Основной принцип свинг трейдинга следующий: вы торгуете акцию в сторону долгосрочной тенденции, но только после того, как произойдет волна сделок против этой тенденции. Если другими словами, то свинг трейдер торгует разворот отката цены акции к основному тренду. Это и есть главная концепция, к которой мы будем возвращаться еще не раз на страницах этого блога.

Например, длинную позицию нужно открывать на восходящем тренде, после того, как была совершенна волна продаж. Мы входим в короткую позицию, если на нисходящем тренде пройдет волна покупок.

Это самый безопасный способ открытия позиции, который мне известен. Если рассмотреть свинг трейдинг, как командную игру, то вы будете играть за команду победителей. Хотите увидеть подтверждение на примере? Смотрите график ниже:

%\includegraphics[scale=0.4]{SwingTrading.png}

Анализируя слева направо, здесь цена акции перешла от восходящего тренда в боковой. Те, что пытаются торговать пробои в этом коридоре, уже дважды потерпели неудачу. Свинг трейдеру нужен тренд для торговли, и он дожидается, наконец, когда цена прорывается значительно далее верх (справа). Ему интересно, подтверждается ли эта восходящая тенденция на большем таймфрейме.

%\includegraphics[scale=0.4]{SwingTrading1.png}

Свинг трейдер открывает недельный график и видит, что акция пребывает на своем двухлетнем максимуме, а с начала года держится мощный восходящий тренд. Он заключает, что бумага явно пребывает в сильной бычьей позиции.

Возвращаемся к первому графику. Свинг трейдеру осталось дождаться на восходящем тренде, когда произойдет волна продаж. Это происходит в зеленой области, где цена понижалась три дня подряд. Теперь свинг трейдер ищет разворотную свечную модель, которая бы позволила войти в позицию.

Что произошло далее с данной акцией, смотрите на графике:

%\includegraphics[scale=0.4]{SwingTrading2.png}

Восходящая тенденция продолжилась и принесла свинг трейдеру хороший доход. Размер прибыли зависит от того, какая у вас стратегия выхода, и как вы управляете своей позицией. Вообще, свинг трейдер держится до конца волны и уходит с рынка. Но когда бумага, как здесь, прорывает важный уровень сопротивления, то сделку можно попридержать до конца восходящего тренда.

Вот это и есть свинг трейдинг. Свинг трейдер открывает позицию, когда есть тренд, и только после волны сделок против этого тренда. Это безопасно, потому что те, кто торгует контртренд, уже совершили свои сделки, и долгосрочная тенденция может продолжаться далее. Если вас заинтересовало это, то читайте далее статьи блога о трейдинге, которые полностью посвящены свинг трейдингу. Будьте успешными!

\subsection{4 фазы рынка, которые должен знать каждый трейдер}

Здравствуйте, дорогие читатели блога о трейдинге. Это первый, вступительный пост из всего курса. И говорить будем о фазах рынка. Когда вы пройдете этот материал, то поймете, как и где на рынке можно зарабатывать. Представленная информация служит базой для понимания стратегии свинг трейдинга.

\subsubsection{Для чего необходимо знать фазы рынка}

Чтобы успешно торговать акциями, для начала вам нужно изучить фазы, которые проходят отдельные бумаги и весь рынок вцелом. Они подскажут, должны ли вы торговать в лонг, шорт или закрыть позиции и оставаться при своих. Как только вы научитесь определять фазы рынка, вам даже не придется задумываться «покупать или продавать». Без всяких вопросов, вы знаете, что делать в данный момент, просто просмотрев графики.

\subsubsection{ Определение фаз рынка}

Ниже на рисунке вы видите схематичное представление того, о чем мы говорим. Это происходит независимо от выбранного таймфрейма: месячные графики, недельные, дневные или внутри-дневные.

%\includegraphics[scale=0.4]{Fazy%20rynka.png}

\paragraph{Первая рыночная фаза}

Еще называется аккумуляции. Это период сразу после длительного
падения. Бумага торговалась вниз, а сейчас выровнялась и идет в
боковом коридоре, формируя базу. Продавцы, которые двигали рынок,
начинают уступать свои позиции покупателям, которые становятся более
агрессивными. Акция просто движется без явного тренда. Как правило,
такие бумаги не «интересны»  и пропускаются через фильтр!

\paragraph{Вторая рыночная фаза}

Это период, когда делаются деньги (смотрите стратегия входа свинг
трейдера). Акция прорывает базу, выходит из бокового канала и
начинается стремительный восходящий тренд. Но происходит интересная
вещь: большинство трейдеров остаётся вне рынка! Почему? Потому, что в
подсознании акция все еще не «интересна». Вспомните, чего учат «умные»
книги -  в боковом диапазоне не стоит торговать. И, следуя правилу,
боясь ошибиться, большая масса откладывает открытие позиции вопреки
рыночной ситуации. Но профессиональные трейдера понимают то лучше. Они
скупают акции и готовы свалить их тем, кто спохватился слишком
поздно. И тогда вступает в действие третья фаза.

\paragraph{Третья рыночная фаза}

Еще называется распределения. После стремительного восходящего тренда
во второй фазе, акция начинает снова торговаться в боковом диапазоне.
Трейдеры, которые отлаживали с открытием позиции, начали на закате
восходящей тенденции входить на рынок. Почему? Они видят, что
упустили, и хотят захватить свой кусок пирога. Кто им продает акции?
Опытные, профессиональные трейдера, которые в это время покидают рынок
(читайте стратегия выхода свинг трейдера). Этот этап очень похож к
первой рыночной фазе. Покупатели и продавцы находятся в равновесии и
бумага скользит в сторону. Рынок готов к переходу на следующий этап.

\paragraph{Четвертая рыночная фаза}

Это нисходящий тренд, которого так боятся те, кто пребывает в длинной позиции (запоздавшие из фазы 3). И вы уже, наверное, догадываетесь, что их ждет? Верно, громаднейшее разочарование. Казалось бы, и фундаментальные данные, по всей вероятности, остаются достаточно хорошими, прогнозы положительными. Все считают, что это всего лишь коррекция. Они продолжают держать акцию ожидая, когда цена опять пойдет вверх. Но, покупка была совершена слишком поздно. И это их величайшая ошибка, которую вы теперь не совершите.

Вот пример:

Эти фазы рынков ценных бумаг проявляются на всех таймфреймах без исключения, будь-то пятиминутный график Microsoft, или недельный Dow.

В целом, старайтесь агрессивно покупать во второй фазе рынка, а продавать в четвертой. В первой и третьей — закрывайте свои позиции и оставайтесь вне рынка.

Здесь есть и свои небольшие хитрости (а как без этого?). Дело в том,
что каждая фаза рынка имеет, так называемые, волны. Но об этом в
следующей статье. Спасибо, что остаётесь с блогом о трейдинге!
Надеюсь, информация была полезной. Обсудите тему фазы рынка в
комментариях. Удачной торговли!

\subsection{Волновая теория Эллиота – базовые знания}

День добрый, читатель блога о трейдинге. Волновая теория Эллиота
является продолжением предыдущей статьи о фазах рынка, только, как бы,
под увеличительной лупой. Она всегда имела своих последователей —
трейдеров-оракулов, которые могут предсказывать движение цен на годы
вперед, а также противников, рассматривающие ее, как гадание на
кофейной гуще. Как бы ни было, предупрежу сразу — не ищите в этом
посте какой-то магической торговой формулы или пресловутого Священного
Грааля. Прочтя эту страницу, вы должны понять, что рынок, как и
человек, в своем непрерывном движении имеет белую и черную  полосы,
которые характеризует волновая теория Эллиота, и которые мы будем
использовать для получения прибыли.

\subsubsection{Волновая теория Эллиота — немножко истории}

Волновую теорию разработал Ральф Эллиот, о которой мир узнал из
монографии «Принципы волн» в 1938 году. Сам автор на ней сравнительно
ничего не заработал (насколько я знаю, он умер в нищете). А кто сделал
на ней колоссальных масштабов бизнес, так это рок-музыкант и просто
хороший человек Р. Пречтер. Основываясь на волновой теории Эллиота,
Пречтер предсказал в конце 70-ых XX века биржевой бум 1982-го и крах
1987-го. Но далее с прогнозами как-то не срослось. Поэтому я
рассматриваю волны Эллиота только, как концепцию движения различных
рынков, но никак с целью прогнозирования. Теперь давайте все же
рассмотрим основные положения волновой теории Эллоита.

\subsubsection{Определение волн Эллиота}

Волны Эллиота – это торговая модель, которая показывает, как движутся рынки. Каждый тренд, независимо восходящий или нисходящий, имеет фазы стремительного движения и коррекции (откатов), что и есть основой волновой теории (что такое откаты  и как их торговать).

Для свинг трейдера (что такое свинг трейдинг) наиболее благоприятной является фаза стремительного движения. Волновая теория Эллиота хорошо укладывается в одну из торговых стратегий: торговля первого отката (смотрите стратегию свинг трейдинга). Вы в этом убедитесь сами.

Сначала, рассмотрите и изучите рисунок ниже…

\paragraph{Волны Эллиота и фазы рынка}

Вторая фаза рынка состоит из пяти волн. На графике они промаркированы от 1 до 5. Это детализированный взгляд на восходящий тренд.

Фаза коррекции обозначается на рисунке, как волны A, B, C. Это
коррекция восходящего тренда. Мы не будем много внимание уделять этой
фазе, но важно знать цельную модель.

\paragraph{Первая волна}

Определяет переломный момент в тенденции и знаменует начало нового
восходящего тренда. Когда вы на графике увидите формирование первой
волны, ожидайте первого отката.

\paragraph{Вторая волна }

Это и есть наш откат! Теперь вы ищете точку входа (больше информации:
стратегия входа свинг трейдера), используя свечной анализ. Это и есть,
в общих чертах, наша стратегия первого отката. Мы входим на рынок в
самом начале восходящего тренда. Но о торговой стратегии поговорим
несколько позже. На это еще будет целый раздел.

\paragraph{Третья волна}

Это самая сильная и продолжительная волна в цикле Эллиота. Вот поэтому мы стараемся войти в рынок до того, как она развернется на полный ход, и желаем прокатится до самого конца.

Волны Эллиота проявляются на всех, без исключения, таймфреймах. Когда
мы находим бумагу на первом откате, хочется, чтобы временные рамки
были побольше. Просто следуем тренду. Стараемся выйти вблизи
верхушки. Почему? Потому что, как раз здесь делается большая часть
денег за меньший период времени.

\paragraph{Четвертая волна}

Является огромным разочарованием для тех, кто вошел в рынок слишком
поздно. Когда темп движения акции замедляется – это является сигналом
окончания лучшей части восходящего тренда или второй фазы рынка.

\paragraph{Пятая волна }

После коррекции четвертой волной, рынок делает последний толчок. Это
движение, как правило, сравнительно с третьей волной, вялое и
непродолжительное. Эта последняя фаза не говорит о силе покупателей, а
скорее о неготовности продавцов (это видно по снижению объемов торгов
на фоне растущей цены – так называемое расхождение направлений
векторов цен и объемов или дивергенция). Это также значит скорое
начало продолжительной коррекции.

\paragraph{Волны A, B, C}

Являются частями нисходящего тренда. Вы наверное заметили, что волна А выглядит, как будто коррекцией пятой волны восходящей тенденции. Но это не так. Если дождаться волны В, то мы увидим, что её вершина ниже предыдущей. Она является откатом для новой нисходящей тенденции, а волна С – это третья нисходящая волна, которая проваливается ниже предыдущих минимумов. Эти этапы можно использовать для открытия короткой позиции.

SPY (март 2009—июль 2010) недельный график. Прослеживается волновой цикл Эллиота, после окончания которого, начинается новый

SPY (2009 — 2012) месячный график. Казалось, что в мае 2012 пятая волна окончилась, но это был лишь внутренний небольшой откат. Обратите внимание, как с каждой волной снижаются объемы.

Это основные принципы, на которых строится волновая теория Эллиота. В книгах по техническому анализу можно найти больше информации, но некоторые моменты могут показаться сложными. Также, не всегда легко точно определить, какая сейчас волна Эллиота заправляет рынком. Главное понимать, что тренды существуют и на них действительно зарабатывают деньги.

Надеюсь, что на блоге о трейдинге вы нашли интересующую вас информацию! Спасибо за внимание. Обсудите тему волновая теория Эллиота в комментариях. Удачной торговли!


Полезно знать - боковой тренд флэт

\subsection{Что такое тренд (тенденция)? Почему торговля по тренду
  наиболее предпочтительна?}

Здравствуйте, читатели блога о трейдинге. Сегодня у нас на порядке дня
архиважная тема – что такое тренд или тенденция (одно и то же). Почему
архиважная? А скажите, сколько раз вы, независимо от вашего периода
увлечения трейдингом, слышали подобные выражения: «тренд – твой друг»
или «не торгуй против тренда». Эти фразы на слуху, читая любую книгу
по торговле ценными бумагами или просматривая форумы. Волей-неволей,
понимаешь, что к этим словам нужно прислушаться и, что самое главное,
понять. На этой странице мы и займемся разбором того, что такое тренд.

\subsubsection{Торговля по тренду – строго рекомендовано }

В многих торговля следом за тенденцией ассоциируется с позиционным трейдингом. На самом деле, любой трейдер, будь-то внутридневной, свинг или позиционный, ищет на графике тренд. Дайте сами себе ответ на вопрос: почему?

Основываясь на уже полученной нами информации о фазах рынка и волновой теории Эллиота, можно авторитетно утверждать, что тенденция – это золотая жила, источник прибыли. Тренд действительно наш друг. Следовательно, умение распознавать на графике тенденцию и её изменения (узнайте, как определить разворот тренда в 80\% случаев) является первостепенной задачей.

Забегая немножко наперед, скажу, что для определения направленности
рынка, мы будем использовать скользящее среднее, что очень упростит
нашу торговлю. А сейчас немножко теории.

\subsection{Что такое тренд – определение и основные характеристики}

Как-то раз на одном чате для трейдеров я прочел следующую переписку (ники выдуманы):

—  Новичок: как определяется восходящий и нисходящий тренд

—  Бывалый: если смотришь на график и видишь, что цена движется к верхнему правому углу экрана – это восходящий тренд. Если к нижнему – нисходящий.

Все гениальное – просто. На этом можно было бы и закончить с определением тенденции, но… Каждый уважающий себя трейдер должен владеть языком теханализа.

Тренд – это вектор, указывающий направление динамики рынка. Состоит
она, как вы уже знаете, из отдельных волн: повышающихся и
понижающихся, которые в свою очередь формируют вершины и впадины. Вы
можете наблюдать три вида тренда: восходящий, нисходящий,
горизонтальный. В литературе последний еще называют, как флэт,
торговый диапазон или боковой коридор. Все эти значения равнозначны.

\subsubsection{Восходящий тренд  }

Помните статью, в которой мы обсуждали фазы рынка? Так вот, вторая
фаза – это и есть восходящая тенденция и характеризуется серией
повышающихся вершин и впадин.

\subsubsection{Нисходящий тренд}

Четвертая рыночная фаза – это нисходящий тренд. Он характеризуется серией понижающихся вершин и впадин.

Ниже представлено схематическое изображение трендов:\\

ВВ — возвышающиеся вершины\\
ВМ — возвышающиеся минимумы\\
ПВ — понижающиеся вершины\\
ПМ — понижающиеся минимумы\\

\subsubsection{Тренд vs. флэт}

Флэт или торговый диапазон – это первая и третья фазы рынка. Уже давно было отмечено, а мы только подтвердим, что рынки пребывают в тренде около 30\% времени. В остальной период они двигаются в каких-либо боковых коридорах без определенного направления. Вот как торговый диапазон выглядит схематически:

Торговый диапазон. Цены движутся без видимого направления. Это значит, что ни продавцы, ни покупатели не пребывают в определенности. Стоит ли вам искать здесь прибыль?

Как мы видим цены движутся совсем как-то неупорядоченно. Конечно, есть торговые стратегии посвященные работе именно в торговом диапазоне рынка. Но зачем усложнять себе жизнь и торговать бумагами, которые «прыгают» вверх-вниз, если можно выбирать такие, которые показывают красивые, надежные тренды. Если стараться заработать деньги в торговом диапазоне, то это прямой путь спустить весь ваш капитал в трубу. Придерживайтесь трендов!

Вот пример акции с сильным восходящем трендом:

Прекрасный пример восходящего тренда от компании Apple.

А эта бумага торгуется в торговом диапазоне:

Цена торгуется в ценовом диапазоне 40-55\$ за 2012. Atlas Air Worldwide Holdings Inc.

Какой график вам больше по душе? Определились? Буду рад, если блог о трейдинге помог вам с необходимой информацией. Спасибо за внимание. Оставляйте свои комментарии по теме что такое тренд. Удачных торгов!

Интересно: риск менеджмент на фондовом рынке

\subsection{Линия тренда, кратко о главном}

Доброго времени суток, друзья, коллеги и читатели блога о
трейдинге. Линия тренда — это один из основных инструментов
технического анализа, который необходим нам для ясного понимания
стратегии свинг трейдинга. И в сегодняшнем посте мы рассмотрим его. И
постараемся не только рассмотреть, но и пролить свет на некоторые
спорные вопросы по нанесению данного инструмента на график. Как бы
тема, линия тренда, не показалась проста, есть некоторые нюансы,
которые вам следует знать.

\subsubsection{Линия тренда FAQ }

Линия тренда или тенденции (trendline) является одним из самых простых технических инструментов. Но, не смотря на это, она очень ценна для трейдера и её использование требует некоторой наработки. Этим инструментом вы будете пользоваться всегда! Поэтому давайте разберем некоторые технические особенности.

Во-первых, проводить линию тренда через максимумы и минимумы баров или через уровни закрытия и открытия? Если вы, как и я, используете свечной график, то предыдущее предложение можно выразить другими словами – вести через тела свечей или тени?

Ответ: так как теханализ не есть точным инструментом, каждый человек видит графики по-своему и поэтому линия тренда  на одном и том же графике у разных трейдеров может отличаться. Так что поэкспериментируйте в этом, своего рода, искусстве. Что до меня, то я отдаю предпочтение наносить её через максимумы и минимумы. Хотя это не есть строгое правило и иногда использую тела свечей для этой цели.

Во-вторых,  что делать, если тенденция стала более крутой и цена больше не касается линии тренда на уровне впадин? Или, когда тенденция замедлилась, стала более пологой и прорвала линию тренда?

На этот вопрос ответ вы найдете далее в тексте.

\paragraph{Для чего используется линия тренда?}

Ну, во-первых, ответ кроется в самом названии – для определения тренда. А, во-вторых, для поиска переломных моментов в существующей тенденции. Для её нанесения необходимо иметь как минимум две точки.

При восходящей тенденции рисуем линию через самые нижние точки (впадины), не позволяя цене пересечь ее. Вот так:

Линия тренда на графике SPY

Это хороший пример. Видите, как цены находят поддержку вблизи линии тренда. Это отличные возможности для открытия позиции. Но рынок не идеален:

На этом графике видно, как линия тренда нарушена, но это не стало причиной изменения тенденции. Она только замедлилась и стала более ровной. То есть, это как бы исключение из правила, рынок движется в том же направлении, но несколько под другим углом. Что делать в этом случае?

В таком случае мы наносим еще одну линию тренда. Вот так:

Окончательный прорыв линии тренда после образования разворотной модели голова-плечи

Как видно с рисунка выше, иногда у вас может быть нарисовано несколько этих инструментов на одном графике. А как тогда определить, изменится ли тенденция на противоположную, если нарушена трендовая линия? Как ответ, читайте эту статью: «Трёхшаговый метод, который в 80\% определяет разворот тренда».  А сейчас скажу, что когда цена приближается к линии тренда, мы ищем разворотные модели, которые могут указать на изменение существующей тенденции. В нашем случае сформировалась модель "голова и плечи".

\paragraph{Вот несколько советов:}
\begin{itemize}
\item Чем большее количество раз цена касается линии тренда, тем она сильнее;
\item Для ее нанесения достаточно двух точек, но дождитесь третьей для подтверждения;
\item При нисходящей тенденции рисуйте линию тренда по вершинам;
\item Чем более она крутая, тем менее надежна;
\item Её прорыв не означает изменение тенденции.
\end{itemize}

\paragraph{Заключение}

Линия тренда является достаточно простым и полезным инструментом. Мы с вами, совместными усилиями, уже практически построили костяк для нашей стратегии. Основные моменты, то есть: фазы рынка, волнова теория Эллиота, уже рассмотрены. Осталось разобраться еще с уровнями поддержки и сопротивления. Эти четыре темы, включая сегодняшнюю, служат базой для любой торговой стратегии. Как бы это не звучало просто и легко, но вы будете еще иметь возможность в этом убедиться. Блог о трейдинге благодарит за внимание. Будьте успешными!


Стоит посмотреть - что такое акции ADR?

\subsection{Трёхшаговый метод, который в 80\% определяет разворот
  тренда}

Добрый день, уважаемые читатели блога о трейдинге. Сегодняшняя тема
была, есть и будет актуальной пока жива мировая экономика. И эта тема
– разворот тренда. Вы, наверное, уже потренировались распознавать на
графиках фазы рынка и волны Эллиота, а также чертить линию
тренда. Почему спрашиваю? Потому что без этого невозможно определить
момент разворота существующего  тренда. А если вы не сможете
определить разворот тренда, то и не сможете открыть позицию в
основании новой тенденции. А если… Короче говоря, вся эта цепочка
приведет не к лучшему результату и соответственно, к похуданию вашего
торгового счета. К тому же, если вы серьёзно решили заняться свинг
трейдингом, то вам крайне необходимо научится хорошо читать графики
(как читать свечной график?). А умение распознавать разворот тренда
или изменение настроя рынка – это краеугольный камень при рассмотрении
любого графика на всех таймфреймах.

\subsubsection{Истоки материала}

В одной из своих книг Victor Sperandeo предлагает три шага для идентификации изменений в тенденции. Эта трёхшаговая стратегия помогает избежать многих потерь и найти отличные возможности для торговли. Она поможет вам распознать изменения настроя игроков рынка, соответственно и разворот тренда, приблизительно в 80\% случаев.

Как вы считаете, стоит это вашего внимания?

\subsubsection{Разворот тренда – определяем в три шага}

Если на графике вы наблюдаете падение цены, которому предшествовал восходящий тренд, то это не значит разворот тенденции. Это может быть коррекция, при которой цена откатывает немножко назад. Рынок, как бы, восстанавливает силы для последующего движения вверх.

Давайте рассмотрим пример рынка, который переходит из восходящего тренда в нисходящий. Запомните следующие три пункта и торгуйте всегда за трендом:
\begin{itemize}
\item     Прорыв линии тренда;
\item     Неудачное тестирование предыдущей вершины;
\item     Цена проваливается ниже предыдущего минимума.
\end{itemize}

Теперь рассмотрим эти три шага к развороту тренда  на примере бумаги,
с которой мы изучали линии тенденции:

\paragraph{Шаг 1}

На этом графике мы видим, что линия тенденции нарушена. Но как вы уже
знаете, это не есть поводом к  развороту тренда. В этой точке мы
становимся внимательными, потому что еще не знаем, как поведет себя
рынок.

\paragraph{Шаг 2}

Вы уже знаете, что акция, которая имеет восходящую тенденцию, делает
высшие вершины и впадины. Когда этого не происходит, возникают
некоторые вопросы и сомнения в господствующей тенденции. Как видите,
это произошло с данной бумагой. Она не установила новый максимум, но и
не упала ниже предыдущего минимума.

\paragraph{Шаг 3}

Теперь у нас есть окончательное подтверждение разворота тренда. Вырисовалась картина, удовлетворяющая определение нисходящей тенденции: понижающиеся минимумы и максимумы.

Этот пример показывает, как восходящий тренд переходит в нисходящий. А что насчет  рынков, которые движутся наоборот?

Думаю, вы уже все поняли сами!

\subsubsection{Разворот тренда FAQ}

Как эта стратегия может помочь в свинг трейдинге?

Во-первых, вы сможете высидеть позицию до конца и заблаговременно выйти из рынка, если он будет подавать признаки разворота. Также можно сразу же открыть позицию в противоположную сторону.

Во-вторых, этот метод дает возможность открывать позицию в самый ранний период зарождения тенденции, когда цена движется наиболее стремительно.

Совместима ли трёхшаговая стратегия со свечным анализом?

Еще как. Если вы ко всему уже вам известному будете активно применять свечной анализ, то это станет большим и прибыльным плюсом к вашей торговли.

Есть ли какие-либо технические инструменты, которые могли бы сделать определение акций в стадии разворота тренда визуально более восприимчивым?

Есть. В нашей стратегии (стратегия свинг трейдинга) мы будем использовать такой. Но об этом в следующей статье: «Как использовать скользящее среднее (moving average)?». Информация этого поста будет считаться не полной, если не скажу, что вы можете использовать пресечение скользящих средних с разными периодами, чтобы найти бумаги, которые находятся в зарождающемся тренде. Вот пример:

Moving average с периодом 10 пересекает сверху вниз 30-периодную. Но, чтобы убедится в новом тренде, нужно подождать, пока цена не уйдет ниже метки.

Ну что, вы нашли полезную информацию по теме разворот тренда на блоге о трейдинге? Тогда до встречи в следующем посте. И, конечно же, удачных и прибыльных вам торгов!

Полезно знать - отличие простых акций от привилегированных акций

\subsection{Как использовать скользящее среднее (moving average)?}

Приветствую вас, дорогие читатели блога о трейдинге. В теме
сегодняшнего поста скользящее среднее или moving average – индикатор,
который существенно облегчит нашу торговлю. Мы изучили уже большую
часть технических моментов, необходимых для понимания стратегии свинг
трейдинга. Зная о фазах рынка, волновой теории Эллиота, линии тренда,
признаках разворота тренда, вам движение рыночных цен уже не должно
казаться хаотичным. Смотря на графики акций, вы начали замечать
некоторые закономерности: здесь я могу купить, а там отлично
заработать. Но согласитесь, вся выше изложенная информация, хоть и
несложная, тяжело вкладывается в сознание одновременно. Скользящее
среднее поможет нам с этими трудностями.

\paragraph{Технические моменты использования скользящих средних}

Вот что я считаю наименее полезной информацией в этом посте, так это разбор в деталях математических и технических аспектов скользящего среднего. Этой информации предостаточно в интернете. А лучше всего прочтите книгу Дж. Мэрфи «Технический анализ фьючерсных рынков».

Что такое скользящее среднее? Можно догадаться из названия, что это индикатор, показывающий среднюю цену за установленный период времени.

Есть несколько видов скользящих средних. Мы будем использовать простое и экспоненциальное. Между ними небольшая разница, но все же есть. Обратите на это внимание, изучая теханализ по Дж. Мэрфи.

Отлично! А чем этот индикатор нам может быть полезен? Скользящее
среднее помогает нам, во-первых, выявить тренд, во-вторых, распознать
изменения в существующей тенденции. То есть, как вы, наверное,
догадались, он является нашей палочкой-выручалочкой, беря во внимание
все выше сказанное.

\subsubsection{Скользящее среднее FAQ}

Можно ли с помощью скользящего среднего прогнозировать движение цены?

Нет. Запомните, этот инструмент отражает текущее состояние цены и не является опережающим индикатором.

Бывает ли стратегии на основании скользящего среднего, что работают на любых рынках?

Нет. Скользящее среднее – один из лучших индикаторов, когда на рынке присутствует тренд и один из худших, если его нет.

А какую цену мы, собственно, усредняем?

При настройке скользящего среднего, кроме всего прочего, у вас будет выбор, какую цену вы хотите усреднить. Это максимум, минимум, открытие, закрытие или даже среднее значение цены диапазона свечи или бара. Если вы, как и я, используете свечной график, то усредняйте цену закрытия. В свечном анализе она наиболее значима.

Может ли скользящее среднее служить уровнем поддержки или сопротивления?

Нет. Это индикатор, а не уровень. Если на графике вы увидели такое совпадение, то поищите в прошлом важные уровни поддержки или сопротивления. Они обязательно будут присутствовать.

Какое оптимальное количество скользящих средних должно быть на графике?

Хочу далее представить цитату из книги Мэрфи «Технический анализ фьючерсных рынков», которая основывается на исследованиях Мерил Линч, опубликованных в 80-ых XX века:

    «… наиболее эффективной, по всей видимости, является комбинация
    двух средних скользящих. В свою очередь, наилучшим вариантом такой
    комбинации будет сочетание двух простых скользящих средних …»

\subsubsection{Использование скользящих средних в свинг трейдинге }

ы будем использовать 10-дневное простое скользящее среднее  (Simple Moving Average) и 30-дневное экспоненциальное (Exponential MA). То есть, одно более быстрое, другое более медленное. Для чего? Когда 10-периодное SMA пересекает 30-периодное EMA, это часто сигнализирует смену тенденции. Просто рассмотрите этот пример:

Пример использования скользящих средних на SPY

На этом графике вы видите, как пересечение линий помогает распознать начало новых трендов. Сначала быстрое скользящее среднее уходит под медленное и тренд идет вниз. Далее 10-периодное SMA возвращается вверх и тенденция переходит в восходящую, которая продолжается полгода.

А теперь правило:

Открывайте длинные позиции, когда 10-дневное простое скользящее
среднее находится выше 30-дневного экспоненциального, а короткие
наоборот, когда 10-дневное скользящее среднее находится ниже
30-дневного. Нет ничего проще, что поможет вам торговать всегда в
правильном направлении!

\paragraph{Рекомендации}

Вот три вещи, которые нужно помнить (для восходящей тенденции; для нисходящей – наоборот):
\begin{itemize}
\item     10-периодный SMA должен находится над 30-периодным EMA;
\item     Чем большее расстояние между линиями, тем сильнее тренд;
\item     Обе скользящие средние должны быть направлены вверх.
\end{itemize}

\paragraph{200-периодное скользящее среднее}

Используется для разграничения территории быков и медведей. Я
использую его только на графиках индексов. Если цена выше его, нужно
рассматривать только покупки, если ниже, только шорты.

\paragraph{Резюме}

Как видите, скользящее среднее дополняет наши технические познания рынков, как: фазы рынка, волны Эллиота, тренды. Также помогает определять развороты тренда. Единственное, где оно может навредить – это торговый диапазон. Но при изучении стратегии свинг трейдинга, вы увидите, что торговый диапазон не такая уж и большая помеха. Блог о трейдинге благодарит за внимание. Обсудите тему скользящее среднее в комментариях. Удачных торгов!

Следующая статья по трейдингу: американская фондовая биржа NYSE

\subsection{Уровни поддержки и сопротивления. Разные подходы}

Доброго здоровья, читатели блога о трейдинге. Уровни поддержки и
сопротивления являются одной из основных тем технического анализа, и
рассматриваются в одной связке с учениями о трендах (что такое тренд,
линия тренда, как определить разворот тренда). Делая заключение своих
исследований, Хокхаймер писал: «Получены эмпирические данные,
убедительно доказывающие, что движение цен на рынках не является
хаотичным». Мы уже в этом убедились сами, изучая тенденции, которые
формируются не прямолинейно, а через вершины и впадины. А что кроется
за этими вершинами и впадинами? Что собой представляют уровни
поддержки и сопротивления?

\paragraph{Что кроется за уровнями поддержки и сопротивления?  }

Зачем трейдера постоянно ищут уровни поддержки и сопротивления на графиках? Чем они так важны?  Дело в том, что они показывают нам основополагающий закон любого рынка – спрос и предложение. Повышенный спрос, возникающий в основаниях после снижения цены, порождают покупатели, что приводит к росту акции и формированию уровня поддержки.

Когда цена вырастает на определенный уровень, покупатели начинают
предлагать свои акции на продажу, чтобы фиксировать прибыль,
порождая, таким образом, повышенное предложение, что ведет к снижению
цены и формированию уровня сопротивления.

\subsubsection{Уровни поддержки и сопротивления FAQ}


де на графике находится уровень поддержки, а где уровень сопротивления?

Независимо от направленности тенденции, уровень поддержки размещен всегда ниже текущей цены (в основании), а уровень сопротивления выше (на вершине). Пример:

Уровень поддержки в основании (область низких рыночных цен), а уровень сопротивления на вершине (область высоких рыночных цен)

Как понимать выражение: уровни сопротивления и поддержки меняются ролями?

Если цена преодолела уровень сопротивления, то в будущем велика вероятность, что он станет поддержкой. Это касается и обратного. Пример:

Цена преодолела уровень сопротивления и использовала его на откате, как уровень поддержки.

Как проводить уровни поддержки и сопротивления: через экстремумы теней или тел свечей?

В свечном анализе тени  рассматриваются, как рыночный шум. Но в свинг трейдинге выделяют понятия как область поддержки и область сопротивления. То есть, мы говорим, что акция находит поддержку в области $19.65 — $20.0. Уровня, как просто линия, не существует. Поэтому, проводите уровни поддержки и сопротивления, как показано на графиках выше, с захватом теней.

Когда уровень поддержки или сопротивления считается преодолен?

Здесь есть несколько вариантов. Я, базируясь на свечном анализе
(узнайте, какие преимущества имеет свечной график), рассматриваю
закрытие текущей свечи над/под уровнем, как пробой. Бывает такое, что
после преодоления уровня, например, на дневном графике, цена
возвращается на следующей день обратно. Но, если посмотреть на
недельный график, то можно увидеть, что лишь тень свечи прошла за
уровень. А это, как упоминалось, не считается пробоем (почему я и не
люблю их торговать). Чем на большем таймфрейме проявляются уровни
поддержки и сопротивления, тем они значимее.

\paragraph{Уровни поддержки и сопротивления и свинг трейдинг}

Уровни поддержки и сопротивления и свинг трейдинг

Детально все это мы будем рассматривать позже при изучении стратегии свинг трейдинга с использованием скользящего среднего. Но, что можно сказать в общих словах? Вы и сами все поняли. Следуя основному закону экономики, покупаем, когда есть спрос (от уровня поддержки) и продаем, когда повышается предложение (от уровня сопротивления). Вот пример:

Области, выделенные зеленым цветом, отвечают покупкам от уровней поддержки.

Это работает и в обратной ситуации, на нисходящем рынке, как здесь:

Области, выделенные зеленым цветом, отвечают продажам от уровней
сопротивления.

\emph{Не забывайте, что уровнями поддержки и сопротивления могут
  служить также гэпы (ценовые разрывы) и круглые значения цены.}

На этом графике уровнем сопротивления, а в дальнейшем поддержкой,
одновременно послужил гэп и отметка в 10\$. Используйте их также в
своей торговли. Только не забывайте основное правило: покупайте, когда
есть спрос и продавайте, когда растет предложение.

\paragraph{Советы}

Значимость уровней поддержки и сопротивления зависит прямо пропорционально от:
\begin{itemize}
\item     Величины таймфрейма
\item     Количества раз тестирования
\item     Значения объемов при тестировании
\item     Продолжительности
\end{itemize}

Буду рад, если вы нашли информацию на блоге о трейдинге полезной для себя. Спасибо за внимание. Оставляйте комментарии по теме уровни поддержки и сопротивления. Удачной торговли!

Полезная информация: лучший индикатор силы тренда

\subsection{Психология трейдинга и дисциплина: как торгуют другие и
  как их переиграть}

Добрый день, читатели блога о трейдинге. Психология трейдинга прекрасно отображается на графике акций, который есть ни что иное, как картина человеческих эмоций. Страх, жадность, надежда и эйфория зачастую уводят цену на акции далеко за рамки ее истинной стоимости. Поэтому мы наблюдаем раздутые мыльные пузыри и следующими за ними обвалы фондового рынка. И чтобы не попадаться на рыночный крючок, важно понимать, что трейдер — это человек, который постоянно ищет преобладающее настроение биржевых игроков, в чем ему помогает психология трейдинга.

Психология трейдинга – это инструмент, который поддается анализу. Она создает на графике модели и паттерны, которые повторяются снова и снова. И пока других трейдеров мучают вопросы:

    Наверное, нужно покупать?
    А может лучше продать?
    Время забирать прибыль?
    Убыток? Нет, надо подождать до безубытка

– вы, как дисциплинированный трейдер, используете их психологических демонов для своей выгоды. Если вы спросите профессионального трейдера, как торговать сложившуюся ситуацию на рынке, то, скорее всего, услышите ответ типа: «Откуда я знаю. Делай то, что говорит тебе твоя торговая стратегия».

В конце концов, люди, которые торгуют на эмоциях и по желанию, открывают позиции в самый неудачный момент: когда эйфория на рынке уже льет через край. Тренд разворачивается в обратную сторону. Потери растут… растут… и растут. Наконец, когда страх потери торгового капитала становится невыносимым, сделка закрывается с большим минусом.

Вы не можете нести большие убытки и надеется стать прибыльным свинг трейдером. Учитесь принимать маленькие потери, чтобы сохранить капитал для большого профита. Поверьте в это!

Рассмотрим пример: недисциплинированный трейдер находится в позиции, которая приносит прибыль. Последние две сделки принесли убытки. Знаете, что он чувствует? Волнение! Эйфорию! «Нужно быстрее зафиксировать профит, чтобы покрыть предыдущие потери».

Я уверен, что вы слышали высказывание успешных спекулянтов: «Урезай убытки, позволь прибылям расти». А что происходит в примере выше? Недисциплинированный трейдер делает все наоборот. Он урезает прибыль!

Весь психологический дискомфорт убирается при помощи торговой стратегии (комплексная стратегия свинг трейдинга), которой вы доверяете, и с которой вам комфортно работать. Составьте план на листе бумаги. Перед тем, как открыть позицию, критерии, по которым она открывается, должны быть также записаны на бумаге. Эти критерии должны отвечать вашему торговому плану.

Этот прием я перенял у известного инвестора Уоррена Баффета. Он
говорит, что если не может записать причины, по которым собирается
покупать акцию, то он и не будет ее покупать.

\subsubsection{Психология трейдинга на графике}

Инвестор ищет хороший бизнес, спекулянт – преобладающий настрой толпы. Технический анализ – это инструмент, который сделает с вас настоящего психолога. Вы смотрите на график и видите не модели и паттерны, а психологический настрой участников рынка. Рассмотрим пример ниже:

Трейдеры, торгующие пробои – бумага покупается в момент прорыва уровня сопротивления. Здесь действует теория «бOльшего глупца» – человек ставит на то, что найдутся люди, которые будут покупать выше его точки входа. Если этого не произойдет, то сформируется ложный пробой.

Начинающие трейдеры – входят в рынок в момент наивысшей эйфории. Если покупают все, куплю и я. Только покупают они те акции, что сейчас сбрасывают им трейдера, торгующие пробои.

Свинг трейдеры – здесь работаем мы. От уровня, на разворотной свече (доджи), на возрастающих объемах. Уровень сопротивления становится теперь поддержкой.

Начинающие трейдеры – снова и снова. Когда объемы достигли максимума, они, эмоционально поднесенные, полагают, что рынок будет расти вечно. Свинг трейдеры спихнут им свои акции.

Такая ситуация повторяется до бесконечности, на всех таймфреймах.

Помните, это всего лишь спекуляция. Наш удел не формулы и расчеты истинной стоимости акции. Как думают другие участники рынка – это наша цель. Откуда я знаю это? Я прошел этот путь: начинающий трейдер, внутридневная торговля пробоев, свинг трейдинг.

Психология трейдинга — это то, что хороший трейдер ищет на графике. Учите не сами торговые паттерны, а то, что стоит за ними. Мы должны сначала научиться дисциплинировать себя, а потом, и только потом, зарабатывать деньги (на тех, кто еще дисциплине не научился). Блог о трейдинге благодарит вас за внимание. Обсудите тему психология трейдинга в комментариях. Торгуйте с самого начала правильно!


Похожая статья: торговля на фондовом рынке с плечом

\subsection{Индикатор объемов и объем торгов: практическое значение и
  применение}

Здравствуйте, дорогие читатели блога о трейдинге. Значение сегодняшней
темы, индикатор объемов и объем торгов, для трейдинга в целом и свинг
трейдинга в частности тяжело приуменьшить. Как говорится, на рынке
есть два объективных фактора: цена и в меньшей мере торговые
объемы. Одни трейдера в большей степени привержены этому индикатору,
другие в меньшей. Есть и такие, которые не пользуются им вообще. В
этом посте мы разберем практическое значение, что дает нам объем
торгов. А как часто использовать индикатор объемов, вы уже решите
сами.

\subsubsection{Что такое объем торгов?}

Сначала определение. Объем торгов – это количество всех сделок, проведенных за определенный период. Тут есть один момент, который нужно понимать (по крайней мере, я не сразу понял). Если вы в течении торгового дня открыли длинную позицию на 100 акций и закрыли её, то ваш личный объем торгов составляет 200 акций за сессию.

Обычно, объемы торгов изображаются в виде столбиковой гистограммы в
нижней части графика и называется индикатор объемов. Как вы поняли
одна вертикальная линия – это сумма открытых позиций (длинных и
коротких) и закрытых. Основное значение торгового объема – это
выявление интереса к бумаге со стороны других участников рынка. Если
акция торгуется на низких объемах, значит внимания к ней мало.

\subsubsection{Индикатор объемов: значение и применение}

Каждое движение цены подтверждается объемом. Он указывает на важность
любых уровней, линий, ценовых моделей. Но, об этом немножко позже. А
сейчас рассмотрим одну из самых важных характеристик рынка…

\paragraph{Ликвидность и объем торгов}

Важное понятие для любого стиля трейдинга. Ликвидность фондового рынка – это показатель того, как быстро вы сможете продать/купить бумагу по рыночной цене.

Если акция торгуется с низкими объемами, значит, вовлечено малое количество трейдеров и нам будет сложнее найти продавца, у которого можно было бы купить и покупателя, которому можно было бы продать. В таких случаях мы говорим, что бумага низколиквидная.

Когда торговая активность высокая, то и рынок будет высоколиквидным.

\paragraph{Уровни поддержки и сопротивления на основании объемов
  торгов}

Как определить значимость разных уровней (как применять уровни
поддержки и сопротивления) на графике? Если вы видите всплеск торговой
активности, когда бумага приближается к уровню поддержки или
сопротивления, значит значимость этих уровней велика; и обратное
верно: значимость уровня тем меньше, чем ниже объемы торгов.

\paragraph{Тренды, ценовые графические модели и индикатор объемов}

Что такое тренд, признаки разворота тренда, как применять линию тренда мы уже рассматривали. Прочтите также мой конспект по моделям свечного анализа, который пригодится вам при разборе стратегии свинг трейдинга:

    Свечной анализ в краткой и доступной форме
    Свечной анализ: свечные модели молот, повешенный, модель поглощения, завеса из темных облаков, просвет в облаках
    Свечной анализ: Свечные модели утренняя звезда, вечерняя звезда, падающая звезда, звезды доджи и перевернутый молот
    Свечной анализ: свечные модели харами, пинцет, захват за пояс, две взлетевшие вороны, контратака.
    Топ 10 самые сильные свечные модели

Давайте рассмотрим небольшой пример:

 Изучите этот график. Какие заключения можно сделать:

    Индикатор объемов повышается вслед за развитием тренда;
    На развороте тренда торговая активность заметно возрастает – всплеск объемов;
    Когда рынок движется в боковом диапазоне, индикатор объемов на минимальных уровнях.

Бывают ситуации, когда движение цены вверх не подтверждается ростом торговых объемов. Это называется дивергенцией и конвергенцией и, как правило, является предвестником смены тенденции.

Выше, хороший пример того, как рынок реагирует на повышение и понижение объема торгов. А теперь посмотрите на график ниже:

Видите, как цена, находясь в боковом канале, торгуется на низких объемах. Но, после прорыва уровня в 30\$, торговая активность возрастает в разы. Запомните, если пробой важного уровня сопровождается ростом индикатора объемов, то цена, скорее всего, продолжит движение в выбранном направлении.

И еще одна рыночная ситуация, изображенная на графике:

Еще один интересный момент: если большому объему торгов отвечает свеча с малым диапазоном, значит, большой покупатель или продавец набирает позицию. Еще это называется чрезмерная торговая активность.

\subsubsection{Заключение}

Объем торгов неразрывно повязан с ценовой динамикой. Часто можно увидеть, как торговая активность возрастает как раз перед значительным движением рынка:

Индикатор объема торгов полезный в некоторых ситуациях и довольно простой в применении. Но, как говорилось ранее, он менее объективный, чем цена и не всегда показательный метод оценки рынка. Поэтому пользуйтесь им без фанатизма. Блог о трейдинге благодарит за внимание. Будьте успешными!


Ссылка по теме: модели свечного анализа

\subsection{Как читать свечной график?}

В этой теме мы будем обсуждать свечной график. А именно: что он собой представляет по определению, какое его преимущество над линейным и столбиковым (баров), и будем потихоньку учиться читать свечной график.

Доброго времени суток, читатели блога о трейдинге. Сразу хочу спросить: какому графику вы отдаете предпочтение? Есть варианты ответов:
\begin{itemize}
\item     Линейный график – анализу доступны цены закрытия
\item     Столбиковый график или график баров – можно анализировать цены открытия, закрытия, максимум и минимум, а также торговый диапазон
\item     Свечной график – все то же, что и бар, только более наглядно!
\end{itemize}

Специально сделал акцент на свечном графике, потому что сам его использую и вам рекомендую. Вот почему: во-первых, легкое восприятие; во-вторых, японские свечи – это лучший метод определения эмоционального настроя участников рынка; в-третьих, как свинг трейдеры, мы хотим открывать позиции впереди других и это обеспечат нам свечной анализ.

Есть еще одно, четвертое, но основное преимущество свечных графиков.Что самое главное на рынке? Верно, это цена или движение цены (ну, и объем торгов, но в значительно меньшей мере). Все остальное – субъективное восприятие каждого отдельного трейдера. И свечной график показывает нам эту динамику движения цены лучше всего. Далее по тексту мы все время будем к этому обращаться, и вы поймете, почему я так говорю.

Очень часто свинг трейдеры увлекаются разными формами теханализа,
забывая о самой главной вещи на графике. Вы можете не использовать для
торговли ничего, кроме свечей, и будете успешным трейдером. Может я
немножко и перегнул, но лично мое мнение, ничто не улучшит вашу
торговлю больше, чем изучение искусства чтения свечных графиков.

\paragraph{Свечной график под микроскопом}

На рынке присутствуют всего два типа людей: покупатели и продавцы. Нам интересно узнать, кто контролирует движение цены сейчас. В этом нам поможет анализ свечи.

Рисунок выше показывает строение свечи. Принцип, как и в бара, только
более наглядно. Свеча имеет тело – область между открытием и
закрытием, и фитили или тени с максимумами и минимумами. Вся свеча –
это диапазон движения цены за определенный период. Когда этот диапазон
закрывается ближе к максимуму, то делаем вывод, что цену двигают
покупатели. Если к минимуму, то продавцы.

\subsubsection{Анализ свечных графиков}

Свечи на графике можно разделить на две категории: с широким диапазоном и узким. Широкодиапазонные свечи говорят нам о высокой волатильности и торговом интересе со стороны участников рынка. Узкодиапазонные, наоборот, о низкой волатильности и малом торговом интересе.

Когда мы видим бычью свечу, которая закрывается на верхнем уровне диапазона, то говорим, что покупатели на данный момент более агрессивные и хотят покупать далее. Это заставляет продавцов продавать по более высокой цене. Как следствие, цена растет.

То, где цена закрывается по отношению к торговому диапазону свечи, говорит нам, кто выиграл битву между быками и медведями. Это наиболее важная вещь, которую нужно знать и понимать, при анализе графика.

Рассмотрите рисунок ниже. Стрелками указаны свечи с широким
диапазоном. А теперь заметьте, как рынок следует за ними. Они, как бы,
указывают нам направление, в котором желает двигаться цена дальше.

\paragraph{Как выглядит свечной график со свечами с узким диапазоном}

Начнем с того, что этот тип подразумевает низкую волатильность. Это период времени, когда бумага интересует сравнительно немногих. Где можно искать данные свечи? Если вспомнить фазы рынка, то в фазе 1 и 3.

Если вы еще раз взгляните на график выше, то сможете заметить, что
такие свечи часто являются предвестниками разворота тренда. Почему?
Потому, что низкая волатильность ведет к высокой и наоборот. Зная это,
имеет смысл открывать позицию при низкой волатильность и закрывать при
высокой.

\paragraph{Как выглядит свечной график со свечами с широким
  диапазоном}

Мы уже знаем, что акции предпочитают двигаться в направлении свечей с
широким диапазоном. Если вы видите такие, значит, скорее всего, на
рынке присутствует тренд. Таким образом, с помощью свечного графика,
можно определить интерес к данной бумаге других участников рынка и
торговать в сторону тенденции, указанной свечами.

\paragraph{Что насчет моделей свечного анализа?}

Вы наверное думаете, что сейчас мы будем говорить о молотах, звездах или доджи. Извините, что разочарую вас. На самом деле, когда вы поймете, что за каждой свечой стоит борьба покупателей и продавцов, то вам просто незачем будет запоминать десятки разных ценовых паттернов свечного анализа.

Рассмотрите следующее…

На рисунке справа, вы видите классическую разворотную модель свечного анализа – молот. Что привило к его возникновению? Бумага открылась и медведи в какой-то точке взяли контроль, толкая цену ниже. Думая, что акция будет продолжать падать, многие трейдеры шортили её.

Но, ближе к концу дня быки взяли реванш и заставили шортистов
закрывать их позиции. Бумага стала достаточно сильна, чтобы закрыться
на вершине диапазона. Очевидно, это бычий сигнал!

\paragraph{Еще несколько слов о свечном графике}

Стоит обратить ваше внимание, что японцы предают большое значение ценам открытия и закрытия, чем максимумам и минимумам.

Есть еще один момент, на который хочу обратить ваше внимание (сам это долго не понимал). На рынке для каждого покупателя должен быть продавец и наоборот. Даже, когда вы видите выраженную бычью свечу и мы говорим, что рынок движут покупатели, то это не означает, что продавцов меньше или они вообще отсутствуют. Чтобы купить, нужно, чтобы кто-то продавал. Согласны?

Свечной график имеет много положительных сторон, из-за чего статья получилась большой. Но я считаю, что это все важно знать и понимать. Блог о трейдинге благодарит вас за внимание. Прокомментируйте ниже тему свечной график. Удачных торгов!


\subsection{Топ 10 лучшие свечные модели}

Свечные модели, которые являются лучшими на мой взгляд — это тема сегодняшнего поста. Бычье поглощение, медвежье поглощение, молот, повешенный, утренняя звезда, вечерняя звезда, дожи, просвет в облаках, завеса из темных облаков — составляющие топ 10 свечные модели.

Приветствую вас, дорогие читатели блога о трейдинге. В этом посте я хотел бы рассказать о своих любимых свечных моделях. Их 10. Все они хорошо себя зарекомендовали при торговле на протяжении десятилетий. Поэтому запомните их и ищите на графике торгуемых акций.

В предыдущей теме мы начали изучать свечной график и его преимущества. И одно из главных преимуществ именно для свинг трейдинга – это возможность открывать позицию, используя свечной анализ, ранее других.

Перед тем, как начать, хочу сказать, что самое важное – понимать, что происходит за завесой свечной модели. Ведь каждая свеча демонстрирует нам борьбу покупателей и продавцов. Секрет вашего успеха зависит от умения определять победителя.

И еще… не бойтесь комбинировать различные варианты теханализа. Например, если на дневном графике вы заметили разворотную свечную модель, то попробуйте найти её подтверждение, используя уровни поддержки и сопротивления, линии тренда (помните посты, в которых мы изучали, что такое тренд и трехшаговый метод определения разворота тренда) и объем торгов, на часовом таймфрейме. Но, это лишь, как небольшая рекомендация.

Следующие свечные модели разделяются на две группы: бычьи и
медвежьи. Это разворотные комбинации, которые мы будем использовать
при торговле откатов (бычьи) и ралли (медвежьи).

\subsubsection{Бычьи свечные модели}

Модели японских свечей: бычье поглощение, молот, утренняя звезда дожи просвет в облаках

Все технические моменты свечного анализа смотрите здесь:

    Кратко и понятно о свечном анализе
    Свечной анализ: свечные модели молот, повешенный, модель поглощения, завеса из темных облаков, просвет в облаках
    Свечной анализ: свечные модели утренняя звезда, вечерняя звезда, падающая звезда, звезды доджи и перевернутый молот
    Свечной анализ: свечные модели харами, пинцет, захват за пояс, две взлетевшие вороны, контратака

Сейчас мы рассмотрим психологию трейдинга, которую представленные
комбинации нам раскрывают. Начнем с первого примера…

\paragraph{Бычье поглощение.}

Моя самая любимая модель. Мы видим две свечи: черную и белую. Первая
говорит о том, что цена все еще за медведями. Но малый диапазон
указывает на угасание интереса и низкую агрессивность продавцов. Белая
же свеча имеет широкий диапазон и, как бы, поглощает первую. Можно
сделать вывод, что спрос превысил предложение и покупатели «скушали»
продавцов. Быки готовы взять контроль над движением цены.

\paragraph{Молот}

В предыдущем посте, когда мы изучали, как читать свечных графиков, эта
модель японского теханализа уже упоминалась. Поэтому, не буду
повторяться.

\paragraph{Утренняя звезда}

Тоже сильный паттерн. Смотрите, за первой черной свечой с длинным
телом (повышенный интерес у продавцов) следует свеча с малым
диапазоном. О чем это говорит? Продавцы уже не настолько
заинтересованы в дальнейшем понижении цены. Последующая белая
широкодиапазонная свеча указывает, что перевес на стороне покупателей,
к которым мы должны присоединится, если хотим победить.

\paragraph{Доджи}

Самая частая, наверное, разворотная свеча. Доджи применяется в ряде
свечных моделей. Но, что скрывает сама свеча? Если на рынке есть
тренд, и появляется свеча с ценами открытия и закрытия на одном
уровне, то это говорит нам лишь об одном – равенстве продавцов и
покупателей. Как дальше себя поведет цена? Нужно дождаться следующей
свечи. Поэтому, доджи рассматривается, как составляющая других
моделей.

\paragraph{Просвет в облаках}

Что-то похожее на первую модель. Первая черная свеча с широким
диапазоном указывает на готовность медведей толкать цену ниже и
ниже. Еще и вторая открывается с гэпом вниз. Казалось бы, сильнейший
медвежий тренд. Но, не там-то было! Практически с самого открытия
второй свечи покупатели гнали цену вверх, закрываясь на верхушке
диапазона, и поглотив больше половины предыдущей черной свечи. Очень
сильный разворотный паттерн.

\subsubsection{Медвежьи свечные модели}

Модели японских свечей: медвежье поглощение, повешенный, вечерняя звезда, дожи, завеса из темных облаков

Как видите, ничего нового. Знакомые бычьи комбинации, только немножко изменены. Попробуйте сами объяснить, что раскрывает каждая фигура.

Если бычьи разворотные модели мы ищем в основании (например, как
завершение отката), то медвежьи на верхушке (например, как завершение
ралли).

\paragraph{Сигнал «рикошет»}

Думаю, вам стоит знать об этой, еще одной комбинации японских свечей. Некоторые трейдеры считают её наиболее сильной. Вы могли слышать её и раньше, как «kicker», что с английского означает: «неожиданный поворот событий».

Модели японских свечей: сигнал «рикошет» или kicker

С рисунка выше вы можете увидеть, что это и вправду неожиданный
поворот событий. Сильный тренд просто ломается после появления свечи
противоположного цвета, с длинным телом, да еще и открывшуюся с гэпом
вверх.

\subsubsection{Следует ли ждать подтверждения при появлении этих свечных моделей?}

Я выскажу лично свое мнение. Нет. Зачем ждать? Чтобы дождаться, когда цена «улетит» в давно указанном направлении? Я отдаю предпочтение открывать позицию около закрытия торговой сессии последний свечи ценовой модели. В это время я уже уверен в том, что сформировалась разворотная комбинация японских свечей.

И еще одна причина, почему не нужно дожидаться подтверждения. Вы же свинг трейдер? И хотите победить? Тогда старайтесь входить одним из первых и принимать участие в окончательном формировании фигуры.

Свечные модели — мощный инструмент в руках практикующего трейдера. Продолжайте изучать материалы на блоге о трейдинге. Комментируйте тему свечные модели ниже и удачных торгов!

Возможно вас заинтересует: анализ линии тренда

\subsection{Схема чтения свечных графиков для свинг трейдера}

Доброго времени суток, читатели блога о трейдинге. Успех спекуляции во многом зависит от умения читать и понимать графики. Чтение свечных графиков – это искусство, изучение которого может занять немало времени. Зачем нам читать графики? Потому что это дает нам представление, куда движутся «большие деньги».

Если вы сумеете проанализировать свечной график и прийти к заключению, стоит ли рисковать своим капиталом в данный момент, то у вас уже заложен крепкий фундамент. Эта статья поможет вам подняться еще на ступеньку выше.

Ключевым отличием начинающего трейдера от профессионала – это умение отличать плеву от зерен на графике. Есть несколько факторов, которые имеют огромную ценность для трейдинга. Анализирую эти факторы, мы можем с большой вероятностью заключить, в каком направлении будет двигаться акция.

Ниже есть список вопросов, который вам следовало бы распечатать в нескольких экземплярах. Когда будете читать графики, то отвечайте на вопросы своих распечаток все как видите. Итак, вот перечень:

    График акции «плавный» или «рваный»?
    Какая фаза рынка на графике акции?
    Какой тренд: восходящий, нисходящий, коридор?
    Какая волна по Эллиоту?
    Сильные уровни поддержки и сопротивления на графике акции?
    Какая сила тренда, если он присутствует?

Как видите, информации немного, и это все то, что мы уже рассматривали
в предыдущих статьях. Теперь давайте пройдемся по каждому пункту на
примере такого вот графика:

\paragraph{«Плавный» или «рваный» график}

Определяется визуально при первом взгляде на график. Таким образом, отсеивается много акций при отборе. Взгляните на рисунок ниже и сравните с нашим примером.

Свечи двигаются хаотично, с частыми ценовыми разрывами между
ними. Тренд отсутствует. Слишком много длинных теней на свечах.

\paragraph{Определяем фазу рынка для нашего примера}

После длительной консолидации, которая длилась до средины января, что
соответствует фазе 1, началась вторая фаза. То есть, сейчас акция
находится в фазе 2.

\paragraph{Какое направление тренда?}

Здесь проблем возникнуть не должно. Если график стремится в правый верхний угол, то тренд восходящий. Основное подтверждение правоты – 10 SMA (скользящее среднее) выше 30 EMA.

По линиям тренда я не торгую, поэтому они и не нанесены на
график. Если хотите их нарисовать, то учтите, что тренд усиливается по
ходу движения. Поэтому одной линией тренда тут не обойдешься.

\paragraph{Считаем волны Эллиота}

Первый откат после смены тенденции является второй волной Элиота. Недавнюю смену тенденции мы определили по пересечению 10 SMA выше 30 EMA. Если вы торгуете первый откат после смены тенденции, то вы торгуете третью волну Эллиота. Она всегда является самой продолжительной и «щедрой» для трейдера.

В нашем случае акция находится на четвертой волне Эллиота, то есть
втором откате. Пока момент разворота на пятую волну Эллиота не
наметился.

\paragraph{Сильные уровни поддержки и сопротивления}


Моя рекомендация: отмечайте на графике все важные уровни поддержки и сопротивления. Это поможет вам с определением смены направления тенденции. Если вы торгуете первый откат после изменения тренда, то очень часто он разворачивается в третью волну Эллиота именно от уровня.

На нашем графике есть достаточно сильный уровень (отмечен), который во время консолидации служил сопротивлением, но после пробоя сработал поддержкой. Он и определил точку разворота тренда в третью волну Эллиота.

Анализируя текущую ситуацию, вблизи формирующегося второго отката мы
не наблюдаем уровня поддержки. Но помните: свеча с широким диапазоном
часто работает как уровень. У нас как раз имеется такая свеча и сейчас
цена торгуется в ее пределах.

\subparagraph{Сила тренда}

Сильный тренд говорит о достаточной его поддержке рыночной публикой. Он менее подвержен ложным движениям. Помните: сила тренда не постоянная величина, она изменяется волнообразно, поэтому, когда значение ее достаточно большое, нужно ждать ослабления в тенденции.

Сила тренда определяется с помощью индикатора ADX (не указан на
графике). Сейчас его значение около 40, что говорит о сильной
тенденции (чем больше, тем лучше). С другой стороны, значение 40
является около максимальным для данной акции.

\paragraph{Итог}

Суммируя все наши наблюдения, можно с большой вероятностью ожидать продолжения восходящего тренда. Но, не забываем, что нет ничего абсолютного.

Итак, мы нацелены на движения вверх. Какие ожидания от него? Тренд после разворота (если он произойдет) перейдет на пятую волну Эллиота. Это самая непродолжительная волна. Причем сила тренда достигла или достигает своих максимальных значений. Вывод: если движение вверх и ожидается, то не нужно рассчитывать на его большой рост.

На этом мы закончим. Нашли для себя что-то полезное? Блог о трейдинге благодарит за внимание. Торгуйте по тренду!


Интересная тема - фигура голова и плечи как сигнал разворота тренда

\subsection{Как читать свечной график, как книгу?}

Приветствую вас, читатели блога о трейдинге. Чтение свечных графика
чем-то напоминает чтение книги. Когда мы видим книгу с интересным
оглавлением, то хочется раскрыть ее на первую страницу и посмотреть,
какие входят в нее главы. Если интерес к ней сохраняется, то мы
переходим к чтению текста. Так же и с чтением свечных
графиков. Читайте объяснение далее в статье.а

\subsubsection{Оглавление}

Это графическая модель, которую вы ищите. Дневной график здесь выступает в роли «большого полотна», на котором производится поиск.

Оглавление, как и в книге, должно заинтересовать вас, рассказать, будет ли интересно читать далее. Так что, установите на своем дневном графике период 6-9 месяцев. Это даст вам достаточно пространства для идентификации моделей.

Вот пример:

Неплохо. Интригующе. Сильный уровень сопротивления по \$33 пробит, и цена закрепилась достаточно высоко. Пересечение скользящих средних указывает на смену направления тренда.

Такое оглавление нас интересует. Интересно посмотреть, как события будут развиваться дальше. Сможет ли уровень сопротивления стать поддержкой.

\subsubsection{Главы}

Это тот же график только несколькими днями позже. Хорошая установка для открытия длинной позиции. Цена откатилась к предыдущему уровню сопротивления и это первый откат после смены тренда и после пробоя важного уровня. Интрига растет. Сможет ли цена укрепится и развернутся в третью волну Эллиота?

Отлично. Но здесь не хватает одного компонента…

\subsubsection{Текст}

Свечи на графике – это как слова в книге. Они показывают, кто побеждает в битве быков и медведей. Посмотрите на продолжения графика нашей акции:

Четыре медвежьи свечи подряд и вдруг появляется бычья. Причем нижняя тень ее опускается далеко за уровень сопротивления, но закрытие происходит выше него. Это хороший знак того, что уровень сопротивления может обратиться поддержкой и покупатели возьмут верх над продавцами.

Давайте теперь смотреть, что случилось далее:

Большинство начинающих трейдеров не рассматривают свечные графики таким образом.

Они фокусируются исключительно на «оглавлении». Или слишком привязываются к «тексту». Такое впечатление, что они страдают туннельным зрением, что дает им возможность концентрироваться только на какой-то одной составляющей графика.

Я уверен, что каждый трейдер проходит такие моменты. Нужно учиться воспринимать график целиком. Чем больше вы найдете на нем подтверждений своей правоты, тем лучше.

Например, на графике акции, который рассматривался в этой статье, мы подтвердили правильность своего суждения о продолжении восходящей тенденции и уровнем сопротивления, который дальше стал уровнем поддержки. И пересечением скользящих средних. И свечным анализом (конечно, в идеале было бы, если сформировался «молот» или «падающая звезда»).

Каждый свечной график показывает вам свою историю. Дело за вами: читать ее или нет. Но, если уж взялись, то обратите внимание на все его компоненты. Блог о трейдинге благодарит вас за внимание. Будьте успешными!


А это вы знаете: индикатор волатильности рынка

\subsection{Вся правда о технических индикаторах}

Добрый день, читатели блога о трейдинге. Есть ли ценность в использовании технических индикаторов? С каким следовало бы поработать в первую очередь? Существует ли индикатор, который лучше других? MACD? Стохастик? RSI? Остановите это безумие! Единственная вещь, которая вас должна интересовать и, которую вам нужно понять, – это движение цены.

Если вы начинающий трейдер, то самую плохую услугу, которую можно
сделать самому себе, – это пытаться торговать акции по техническим
индикаторам. Само движение цены – вот что вас должно интересовать в
трейдинге. Это основа, на которой вам следует сосредоточить все свое
внимание.  Так что отложите пока OBV, CCI, PPO и просто сфокусируйтесь
на графике.

\subsubsection{Опережающие и отстающие индикаторы}

Все технические индикаторы в целом делятся на две категории: опережающие индикаторы и отстающие индикаторы. Опережающие индикаторы, например стохастик, подразумевается, что идут впереди движения цены (прогнозируют следующий шаг). Отстающие индикаторы, например скользящее среднее, следуют за движением цены.

Это так, как написано в книгах и как подразумевается. Но в реальности все технические индикаторы являются отстающими индикаторами по одной простой причине: не может график рисоваться, пока не произойдет движение цены. Это как если бы колеса ехали впереди повозки.

Запомните, в основе работы всех технических индикаторов лежат цены открытия и закрытия, максимумов и минимумов, а также значения объемов торгов. Эта информация берется из движения цены акции, которое уже совершилось, и показывается на вашем графике в виде RSI,MACD и др. Другими словами, любой индикатор не может вам сказать больше того, что уже сказал график!

Использовать технический индикатор – это как смотреть в бинокль на
любимую рок группу, стоя на их концерте в фан-зоне, перед
сценой. Зачем такое делать? Отложите бинокль и просто смотрите на
сцену! То же и с графиками. Просто смотрите на движение цены.

\subsubsection{Сначала научитесь интерпретировать цену}

Хорошо, теперь, когда вы знаете всю правду о технических индикаторах, можете расслабиться. Надеюсь, вы прекратите искать тот всемогущий индикатор, который решил бы все ваши проблемы в трейдинге.

Так что нужно искать на графике? Хороший вопрос, не так ли? Главную вещь, которую вам необходимо попытаться определить – это психология других трейдеров. Помните, инвестор ищет бизнес, спекулянт – настрой толпы.

Вы пытаетесь определить, где они собираются покупать, а где продавать. Пытайтесь прочесть их мысли, влезть в их головы. Вы хотите знать их настрой. Они возбуждены, нервозны, напуганы или равнодушны?

Каждая акция, на любом таймфрейме, поочередно отображает эти четыре эмоции. Акция прорывается из консолидации (возбуждение), импульс замедляется (нервозность), трейдеры начинают продавать (страх), и когда равновесие между покупателями и продавцами выравнивается, наступает равнодушие.

Как спекулянт, пытайтесь определить точку, в которой одно
эмоциональное состояние перерастает в другое. Лучший инструмент для
этой цели – свечной анализ (за что я его так люблю). Свечные модели
дадут вам сигналы намного быстрее любого технического индикатора!

\subsubsection{Для чего тогда технические индикаторы?}

Индикаторы можно и нужно использовать в трейдинге. Только делать это нужно правильным путем. Индикатор, по определению, создан для индикации. То есть, используйте, например RSI, для индикации возможной точки разворота. Дальше забудьте о нем и сфокусируйтесь только на движении цены акции.

Очень часто я вижу, как трейдеры открывают позицию только потому, что индикатор перекуплен или перепродан. Акция может быть перекупленной или перепроданной очень долго. Убедитесь в этом сами, наблюдая любой тренд.

Ищите дивергенции. Если технические индикаторы имеют полезные функции, то возможность идентифицировать время, когда цена расходится с их показателями, является одной из наиболее полезных. Это может служить сигналом приближения точки разворота. Как обычно, следите за свечами (ценой) для подтверждения.

Используйте подходящие индикаторы для работы. Для анализа тренда применяйте индикаторы следующие за трендом, например скользящее среднее. Для торговли в ценовых коридорах, воспользуйтесь осциллятором, например RSI.

Помните, нет никакой необходимости использовать какой-либо индикатор при торговле акциями. Наоборот, когда вы только учитесь трейдингу, тренируйтесь на чистых графиках до тех пор, пока не прейдет ясное понимание того, как интерпретировать движение цены.

Практика делает ученика мастером.

Даю вам прием, который использую сам. Распечатайте несколько графиков случайных акций. Не устанавливайте никаких индикаторов! Не применяйте даже скользящие средние или объемы на них.

Сделайте для себя чашку кофе, сядьте в свое рабочее кресло, поерзайте по нему, найдя удобное положение, и примитесь за рассматривание каждой свечи на графике.

Ищите уровни поддержки и сопротивления, линии тренда и эмоциональные экстремумы. Можете рисовать на них свои заметки. Пытайтесь читать мысли других трейдеров, лежащих в основе движения цены. Наиболее важно, сможете ли вы предугадать их следующий шаг?

Я полностью согласен с выражением «технические индикаторы созданы для начинающих, которые не в состоянии понять цену». Практикуйтесь. Проанализировав тысячи графиков, изменения будут на лицо. Блог о трейдинге благодарит за внимание. Успехов!

Еще по теме: что означает волатильность?

\subsection{Священный Грааль в трейдинге существует!}

Привет, читатели блога о трейдинге. Ищите Священный Грааль в трейдинге? Ищите Священный Грааль в техническом анализе? Большинство будет говорить вам, что его не существует. Но он есть. Существует ли он в какой-то особенной торговой стратегии? Нет. Может это волшебный технический индикатор? Нет. Тогда это должен быть какой-то графический чудо-паттерн? Даже близко не то.

И все же он существует – прямо перед вашим носом.

Вся проблема в том, что восприятие трейдинга большинством людей искажено. Они пребывают в постоянном поиске каких-то чудесных систем, стратегий, индикаторов, которые бы все время указывали на прибыльные сделки. Такого не случится никогда. Нет ничего такого, что могло бы помочь в достижении этой цели, потому что существует много случайных факторов, которые непредвиденно действуют на рынок.

Однако есть два разных фактора, которые, если их скомбинировать, помогут нам достичь того, что определенно можно называть Священным Граалем в трейдинге. Готовы?

Мы можем разделить его на две части: технический анализ и риск
менеджмент.

\subsubsection{Технический анализ}

Кто самые прибыльные трейдеры в мире? Никто? Самые прибыльные трейдеры в мире – это маркет мейкеры и специалисты. С момента образования рынка они постоянно зарабатывают деньги. Так почему они так успешны?

Их успех кроется в том, что они торгуют против рыночной толпы. Они все время по другую строну ордеров. Когда вы покупаете, они продают вам. Когда вы продаете, они покупают ваши акции. Они знают, что вы собираетесь делать дальше. Как? А ведь человеческая психология совершенно не изменилась.

Сотни тысяч трейдеров торговало до вас и делали все те же ошибки снова и снова. Эти начинающие трейдеры продавали возле поддержки и покупали возле сопротивления, а умные трейдеры зарабатывали на них.

Священный Грааль технического анализа – это, в первую очередь, идентификация начинающих трейдеров, потому что именно у них вы будете «отбирать» деньги. Чтобы регулярно зарабатывать, торгуя акциями, вам необходимо, чтобы эти трейдеры регулярно теряли деньги для вас! Без них никто не смог бы получать прибыль на рынке вообще, да и самого спекулятивного рынка не существовало бы.

Спекулянт зарабатывает на ошибках других. Если вы потеряли деньги в сделке, значит кто-то заработал на вас.

Большие инвестиционные учреждения делают то же, только в бОльших масштабах. Они потихоньку скупают акции во время медвежьего рынка и затем сбрасывают их рыночной публике на протяжении бычьего. Они получают прибыль с начинающих инвесторов.

Так как же нам идентифицировать начинающих трейдеров? Просто посмотрите на любой график и вы увидите их. Они продают тогда, когда все значительные продажи уже состоялись, возле уровня поддержки. Они покупают тогда, когда все значительные покупки уже состоялись, возле уровня сопротивления.

Вы должны покупать тогда, когда начинающие трейдера продают. Продавайте, когда начинающие трейдеры покупают. Вы делаете в точности то же, что и маркет мейкер, и специалист, день ото дня. Вы торгуете против рыночной толпы, и все же эта толпа помогает вам стать одним из немногих трейдеров, которые действительно успешны в краткосрочном трейдинге.

Разве не ирония: начинающие трейдеры ищут Священный Грааль в
техническом анализе, чтобы со временем понять, что сами являлись этим
Граалем, только для других, более опытных трейдеров?

\subsubsection{Риск менеджмент}

Вначале статьи, я говорил, что «существует много случайных факторов, которые непредвиденно действуют на рынок». Вот поэтому невозможно избежать убыточных сделок в трейдинге. Выходят плохие новости, возникают глобальные кризисы, террористические атаки… перечислять можно долго.

Таким образом, чтобы уберечь свой капитал от этих событий, вам необходима хорошая стратегия риск менеджмента, которая состоит с трех частей: самоуправление, управление капиталом, управление позицией.

Наибольший ваш враг на фондовом рынке – это, несомненно, вы сами. Никто и ничто более. Вы человек и у вас есть эмоции, которые будут мешать торговать, если не научится ими управлять. Вы можете быть возбуждены, нервны, напуганы, жадны, счастливы и разочарованы. Все эти эмоции могут привести к потери денег.

Как защититься от самого себя? Согласитесь, если человек изо дня в день работает по привычной схеме, то неожиданностей у него возникает немного. То есть, вы должны работать по стратегии или торговому плану. Но плана недостаточно. Нужно иметь незыблемую дисциплину, чтобы следовать своей стратегии. Учитесь искусству самоуправления и станьте хозяином своих эмоций.

Даже если трейдер следует своему торговому плану, он все равно может потерять деньги. Сколько вы готовы потерять в убыточной сделке? В этом поможет разобраться управление капиталом. У вас должна быть конкретная цифра, желательно в процентах от торгового капитала. Есть широко употребляемое правило: убыток не должен превышать 2% от торгового капитала.

Рискуя такой небольшой суммой, вам нестрашна убыточная сделка. Это всего лишь один трейд. Будет еще много других. Вы временно потеряли небольшую сумму денег. Вернете их в следующий раз. Так что, расслабьтесь!

Как будете управлять своей открытой позицией? Используйте трейлинг стоп! Если вы свинг трейдер, то пользуйтесь близкими стопами. Торгуя тренд, нужно применять более отдаленный стоп.

Определите для себя, как вы собираетесь выходить из позиции, и
следуйте этому указанию. Помните, ваша задача полностью убрать эмоции
из трейдинга. Трейлинг стоп – это лучший путь для достижения этой
цели.

\subsubsection{Священный Грааль}

Вот он. Это Священный Грааль в торговле акциями. Каждая прибыльная торговая система или стратегия, так или иначе, зарабатывает на слабых и дезинформированных – начинающих трейдерах. За этими стратегиями стоят люди, которые профессионально владеют искусством управления: самоуправлением, управлением капиталом и управлением позицией.

Имеются у вас все эти составные, изложены в письменном виде, с четкостью и пониманием? Большинство людей не имеют психологической установки, чтобы стать трейдером. Не нами сказано, что 80\% трейдеров терпят неудачу и либо теряют все деньги, либо уходят в первый же год торговли.

В этом и весь смысл. Фондовый рынок создан, чтобы большинство потерпело неудачу. Это единственный путь получения прибыли для немногих оставшихся. Блог о трейдинге благодарит за внимание. Успехов!


Это познавательно: как определить силу тренда?

\subsection{Как выбрать фондового брокера?}


Здравствуйте, читатели блога о трейдинге. Как выбрать брокера, который
бы соответствовал всем вашим торговым принципам? В этом посте мы
разберем, какие есть варианты фондовых брокеров. Выбор лучшего брокера
для удовлетворения своих торговых целей является основным решением,
которое не относится непосредственно к торговле, когда вы торгуете
акциями. Поэтому возьмите себе на заметку все советы по теме, как
выбрать брокера.

\paragraph{Типы фондовых брокеров}

Есть два основных типа фондовых брокеров: брокеры полного профиля и дисконтные брокеры.

О разнице между ними мы и поговорим далее.

\subsubsection{Брокеры полного профиля}

Брокеры полного профиля, например Меррилл Линч, предоставляют клиентам
полный набор услуг, включая управление денежными средствами,
консультирование, исследования по инвестированию и т.д. Эти ребята
дадут вам совет по вложению капитала. Выбор этого типа фондового
брокера для долгосрочного инвестирования будет вашим лучшим решением,
если вам нужна помощь в этой области.

\subsubsection{Дисконтные брокеры}

Дисконтные брокеры, как например Ameritrade, берут за свои услуги значительно меньшие комиссионные и гонорары, но они не предложат вам никаких дополнительных услуг в отношении, какую акцию покупать, а какую продавать, или план пенсионных сбережений, если вы собираетесь через 10 лет уйти на покой.

Другими словами, вам дается возможность торговать акциями, а как вы это будете делать, забота ваша. Но не напрягайтесь! Изучив материалы блога о трейдинге, вы не будете нуждаться в чьих-либо советах!

Итак, для свинг трейдинга и дейтрейдинга лучшим выбором станет дисконтный фондовый брокер.

Подождите еще минутку! Перед тем, как вы побегите открывать счет, вам
необходимо еще узнать о двух типах онлайн брокеров. Вы же собираетесь
торговать через интернет, не так ли?

\paragraph{Онлайн брокер на веб основе}

С типичным онлайн брокером на веб основе (например, Ameritrade) вы можете открывать позицию и устанавливать ордера на покупку или продажу прямо из своего браузера. Вам не нужно устанавливать каких-то приложений. Вместо этого, вам просто необходимо зарегистрироваться и зайти на сайт вашего брокера, выбрать экран ордера и ввести свой ордер.

В реальности это выглядит, как если бы вы им послали электронное письмо с покупающим или продающим ордером. В принципе все достаточно просто и понятно!

Хорошо, если все ясно, то давайте смотреть на недостатки таких онлайн
брокеров. Есть две проблемы, которые известны, как «плата за поток
ордеров» и «внутреннее исполнение».

\subsubsection{Плата за поток ордеров}

Когда вы исполняете трейд, ваш брокер может направить ордер к третьей
стороне, за что он получает откат (в денежной форме,
конечно). Очевидно, что результатом для вас может стать неполучение
той цены, которая ожидалась, из-за значительного удлинения времени
исполнения ордера.

\subsubsection{Внутренне исполнение}

Это то, что может вас реально порадовать. Онлайн брокер на веб основе может исполнить ваш ордер из своих личных запасов акций и, естественно, заработать на этом. Для них это хорошо тем, что могут избежать оплаты взносов ECN. Для вас – скоростью. В действительности, ваш ордер может даже вообще не попасть на рынок. Вместо этого брокер исполняет его из своих личных средств.

Так что помните, ваш брокер может зарабатывать на вас не только за
счет комиссионных. Почему? Потому что они контролируют оборачиваемость
ордеров.

\paragraph{Онлайн брокер с прямым доступом}

Онлайн брокер с прямым доступом позволяет вам торговать напрямую с
маркет мейкером, специалистом, или ECN через специальные
приложения. Третья сторона здесь автоматически исключается, и также
нет внутреннего исполнения, о котором говорилось ранее. Вы
контролируете оборачиваемость ваших ордеров на различных
биржах. Только вы и рынок!

\subsubsection{Специальные приложения}

Чтобы торговать через онлайн брокера с прямым доступом, вам нужно
специальное программное обеспечение. Здесь есть два варианта: либо
брокер обеспечивает вас этими приложениями, либо вы скачиваете и
устанавливаете их с других ресурсов. Так как это все платно,
проверяйте наличие этой возможности у вашего брокера.

\subsubsection{Комиссионные ECN}

В дополнение к комиссионным, которые вы платите за каждый трейд, с вас могут сниматься платежи за пользование ECN (электронная коммуникационная сеть). Это система прямого доступа, через которую ордер сразу попадает на рынок от имени клиента без участия брокера.

Дополнительно вы можете платить за рыночную информацию. Она часто включается в цену программного обеспечения для торговли.

Если кратко, чтобы знать все свои расходы, перед тем, как заключить
контракт с брокером, поинтересуйтесь списком всех денежных сборов,
которые у них приняты.

\paragraph{Как выбрать брокера для свинг трейдинга?}

Давайте в заключение определим, какой тип фондового брокера лучше выбрать для свинг трейдинга? Хорошо, вы, разумеется, хотели бы работать с дисконтным брокером. Это само собой. На начальных этапах трейдинга вы можете использовать и онлайн брокера на веб основе, но в некоторой точке обязательно возникнет необходимость в максимальном контроле своих ордеров.

Тогда у вас возникнет желание перейти к онлайн брокеру с прямым доступом.

Поиск хорошего фондового брокера достаточно времязатратный процесс. Если у вас широкий спектр потребностей, то рассматривать нужно много факторов. Нужно потратить также некоторое время для тестирования разных торговых платформ. Главное, чтобы за потраченные деньги вы получили хорошее качество.

Если вы хорошо разобрались в основах свинг трейдинга, то переходите к странице, посвященной стратегии свинг трейдинга. Блог о трейдинге благодарит за внимание. Оставляйте свои комментарии по теме как выбрать брокера. Удачного выбора!


Полезная информация: биржа ценных бумаг в Украине

\section{Свинг трейдинг: стратегия}

Свинг трейдинг — это торговый стиль, который имеет множество стратегий. В этом обучающем разделе вы узнаете об одной из них. Главный момент здесь — торговать, когда цена даст небольшой, обратный откат после стремительного движения. Такой подход дает возможность войти в самом начале нового движения и снижает риски, связанные с ложными колебаниями цены. Читайте более подробно, как производится свинг трейдинг, в следующих статьях.

\subsection{Откат на тренде как модель и принцип торгов}

В этом посте мы разберем, что собой представляют откаты на трендах. Также рассмотрим преимущества торговли на откатах.

Приветствую вас, читатели блога о трейдинге. Мы начинаем изучение стратегии свинг трейдинга. Зная основы технического анализа: что такое тренд, линия тренда, фазы рынка, волновая теория Эллиота – надо научиться их применять. Главный вопрос, который всех волнует: «Где и когда открывать позиции, чтобы они приносили прибыль».

Наверное, вы уже слышали, что покупать нужно то, что растет в цене, а
продавать то, что падает. Но, я хочу предложить вам абсолютно
противоположный вариант. Это торговля на откатах.

\subsubsection{Что такое откаты на трендах?}

Базируясь на волновой теории Эллиота, мы знаем, что тренды не идеальны. Всегда имеются краткосрочные периодичные движения цены в противоположную сторону. Это и есть откат. Он может быть разной продолжительности и величины. Трейдеры образно говорят, что в период откатов на тренде, рынок набирается сил для последующего рывка.

Давайте посмотрим на график ниже на котором изображен восходящий тренд. Как видите, после роста цены, идет небольшой спад или коррекция. Чаще всего основание отката определяется возле уровня поддержки или уровней Фибоначчи.

Вы, наверное, слышали слово «ралли» применимое к трейдингу. Понятными
для нас словами – это оживление или восстановление рынка. То есть, это
откат на нисходящем тренде. Этот термин больше используется в
англоязычной литературе. Мы понятие «откат» будем применять, как для
нисходящей тенденции, так и для восходящей.

\subsubsection{Предпосылки к торговле на откатах}

В книге Ларри Коннорса «Краткосрочные торговые стратегии, что работают» можно найти интересные статистические данные. Они взяли исторические показатели S\&P 500 за 13 лет. Исследовалось три способа открытия позиции: покупка при пробое 10-дневного максимума; короткая позиция при пробое 10-дневного максимума  (то есть противоположно предыдущему условию); покупка при пробое 10-дневного минимума.

Как видите, была взята стандартная стратегия торговли пробоев (первая) и две другие, которые, в общим-то, с первого взгляда не очень логичны: покупка при падении цены и продажа при росте. Но логика полученных результатов говорит о другом. Деньги зарабатывались в двух последних случаях, когда покупали минимумы, а продавали максимумы. А торговля прорывов (в нашем случае 10-дневных) принесла убыток.

Конечно, сейчас я не упомянул о таких важных вещах, как проскальзывание, стопы, размер торгового счета и другое. Это вы сами можете прочитать и проанализировать в книге Коннорса. Но, факт прибыльности стратегии торговли на откатах – в представленном выше примере, это два нелогичных способа покупки на пробое 10-дневного минимума и шорт при пробое 10-дневного максимума – остается фактом.

Поэтому, задумайтесь серьезно над торговлей откатов.

\subsubsection{Первый откат}

Здесь все говорит само за себя. Это первый откат после смены тенденции. Как определить изменение тренда мы разбирали в этой статья «Трехшаговый метод, который в 80\% случаев определяет разворот тренда».

Но для большей удобности и скорости анализа графиков предлагаю применять скользящее среднее. Если 10-периодное простое скользящее среднее пересекает 30-периодное экспоненциальное снизу вверх, то это означает начало восходящего тренда. Если наоборот – сверху вниз – то нисходящего. Вот пример:

Это наиболее надежный способ открытия позиции, тем более, что в этой
зоне, вероятнее всего, на рынок поступают институциональные
деньги. Если вы торгуете по одному паттерну, то это должен быть он!
Прекрасная возможность войти на рынок в начале тренда в точке с
наименьшим риском и взять частичную прибыль или все движение
цены. Чего еще можно желать?

\subsubsection{Первый откат после пробоя}

Вот еще одна модель, стоящая внимания – первый откат после пробоя.

Если вы видите, что акция торгуется в боковом диапазоне, формируя торговую фигуру, и цена прорывает ее, то ищите точку входа при первом откате к уровню поддержки. Это также дает вам мало рискованную позицию, которая может перерасти в продолжительный тренд.

Многие трейдеры-новички торгуют пробои. Но это далеко не мало рискованное вложение капитала (иногда по стопам выкидает несколько раз подряд). Позвольте рынку самому определится с направлением и присоединяйтесь к победителю в самой безопасной области – на откате.

Спасибо за внимание. Оставайтесь с блогом  о трейдинге. Не забудьте оставлять комментарии. Удачных торгов!


Думаете, что бы еще полезненького почитать? - виды заявок в биржевой
торговле

\subsection{Как торговать откаты, используя активную зону трейдера?}

Приветствую вас, уважаемые читатели блога о трейдинге. Вы продолжаете изучать рабочую и готовую к применению стратегию свинг трейдинга. Надеюсь, что материал предыдущей статьи был изложен доступно и мы готовы сегодня рассмотреть, как торговать откаты.

Как свинг трейдеры, мы определились, что покупать нужно после того,
как волна продаж была совершенна, а продавать, когда прошла волна
краткосрочных покупок. Другими словами, торговать откаты. В этом есть
смысл, не так ли? А как определить момент разворота или момент
возвращения тенденции к основному направлению? Для этого мы и будем
использовать активную зону трейдера (далее в тексте АЗТ).

\subsubsection{Что такое активная зона трейдера и как ее легко
  определить}

АЗТ – это область покупок и продаж на графике, которой свинг трейдеры пользуются для выявления промежуточных (внутритрендовых) разворотов в тенденции или коррекций. Вот пример:

Для того, чтоб активная зона сформировалась, нам необходимы две скользящие средние. Если вы были внимательными, то должны были заметить, что мы уже их применяли для определения изменения в основной тенденции в предыдущем посте. Как выявить разворот тренда только лишь графическим методом читайте здесь «Трехшаговый метод, который в 80% определяет разворот тренда».

Итак, строем на своих графиках две скользящие средние:

    Одна, простая 10-периодная (Simple Moving Average)
    Другая, экспоненциальная 30-периодная (Exponential Moving Average)...

Пространство, ограниченное двумя скользящими средними, мы и называем активная зона трейдера. Когда цена попадает в АЗТ – это откат как раз той величины и продолжительности, что нам нужны.

Почему именно 10- и 30-периодные мувинги? Все просто: если цена не достигает 10-периодной скользящей средней, то наличие отката вообще не рассматривается. Если цена опускается ниже 30-периодной, то велика вероятность либо затяжной коррекции, либо разворота тренда.

Важно заметить, что не имеет значения, будете ли вы использовать
Simple MA или Exponential MA. Между ними существует небольшая разница,
поэтому не стоит зацикливаться на этом. Просто пользуйтесь этими
линиями для выявления торговых возможностей. Отдельно о входах в
позицию и выходах мы еще поговорим в других статьях.

\paragraph{В чем особенность активной зоны?}

Было замечено, что именно в АЗТ происходит большинство разворотов. А это то, что нам нужно, чтобы торговать откаты. К тому же, простые настройки (два мувинга) фокусируют наше внимание на нужной области графика и помогают, таким образом, быстро идентифицировать бумаги с нужными нам качествами для торговли.

Но, стоит сказать, мы не рассматриваем скользящее среднее, как
указатель для открытия позиции. Только, как область, где откат должен
развернутся в сторону основной тенденции. И когда это происходит, ищем
уровни поддержки и сопротивления, линию тренда, свечные модели и
т.д. То есть, разыскиваем несколько подтверждающих сигналов в сторону
тренда.

\paragraph{Применимость АЗТ}

Я более чем уверен, что многие люди воспринимают полученную информацию (любую) осторожно, с недоверием. Им нужны «серьёзные» подтверждения.

Похожую стратегию давал Герчик А.М. в книге «Биржевые секреты» для внутридневной торговли (таймфрейм дневной). Только используется одно скользящее среднее с периодом 20. Позиция открывается при пересечении цены с мувингом.

Как мне кажется, торговая зона более применима. Здесь учитывается больше факторов. Если вы со мной согласны, то дальше читайте на блоге о трединге о стратегии входа свинг трейдера. Спасибо за внимание. Оставляйте комментарии. Удачных торгов!


Ссылка по теме: стоп лосс ордера

\subsection{Стратегия входа свинг трейдера}

Приветствую вас, дорогие читатели блога о трейдинге. Торговля акциями – это опасное занятие для вашего капитала, пока вы не выработаете стратегию. В этом посте мы рассмотрим, как входить на рынок, используя стратегию свинг трейдинга, с наименьшим риском для ваших денег.

Стратегия входа свинг трейдера – важная часть торговли. Это время, когда вы рискуете частью своего торгового капитала. Мы разобрались, что ищем акции, которые дали откат в активную зону трейдера. Чтобы открыть позицию нам нужно знать момент, когда цена развернется в сторону основного тренда. Тогда мы откроем сделку в самом основании.

Здесь мы изучим базовую графическую модель для входа на рынок. Когда
вы будете свободно торговать его, рассмотрите другие разворотные
модели, например, почитайте "Альтернативные стратегии входа для
агрессивного свинг трейдинга". Первым делом, вы должны научиться
распознавать точку разворота. Что это? Это графическая модель, что
состоит из трех свечей. Для входа в лонг, ищем точку разворота в
основании, для открытия короткой позиции, ищем точку разворота на
вершине.

\subsubsection{Точки разворота}

Для точки разворота в основании, первая свеча делает минимум, вторая – более низкий минимум, а третья – более высокий минимум. Последняя свеча говорит нам о том, что продавцы выдохлись и бумага, скорее всего, развернется вверх. Главный момент: позиция открывается, когда последняя свеча закрывается выше максимума предыдущей.

Для точки разворота на вершине, первая делает максимум, вторая – более высокий максимум, а третья – более низкий максимум. Итог, покупки прекратились, ждем разворота вниз. Главный момент: позиция открывается, когда последняя свеча закрывается ниже минимума предыдущей.

Рисунок поможет вам лучше разобраться в этих моделях:

А теперь суммируем наши знания. Для входа в лонг, мы ищем акцию, которая имеет восходящий тренд (10 SMA выше 30 EMA) с откатом в активную зону (АЗТ) и точку разворота в основании. Пример:

На графике видны все критерии, которые мы задали. Наша задача, купить акцию на третьей свече около закрытия торговой сессии, когда будем уверены, что она закроется выше максимума второй.

Теперь рассмотрим пример бумаги, которую мы бы хотели шортить. Что искать? Акция должна иметь нисходящий тренд (10 SMA ниже 30 EMA) с откатом в активную зону (АЗТ) и точку разворота на вершине. Как здесь:

Наша задача отвечает предыдущей: открыть короткую позицию около
закрытия третьей свечи, когда будем уверены, что она закроется ниже
минимума второй.

\paragraph{Несколько советов}

Хочу обратить ваше внимание еще на несколько моментов. Во-первых, как вы, наверное, заметили на двух предыдущих графиках, откаты состоят из трех повышающихся белых свеч в первом случае и трех понижающихся черных свеч во втором. Так вот, на заметку: откаты не обязательно должны состоять из трех свечей. Их может быть и больше. Главное – это правильно идентифицировать окончание отката (коррекции), то есть определить точку разворота.

Второе, что хотелось бы сказать, – покупать можно, когда третья свеча в разворотной модели поднимется выше максимума второй. Для шорта – наоборот. Но, все же, мой совет: покупайте/продавайте ближе к закрытию, когда будете уверены, что акция закроется выше/ниже максимума/минимума предыдущей свечи (помните, в свечном анализе большое значение предается ценам закрытия и открытия).

И третье, когда вы нашли бумагу, отвечающую всем необходимым критериям, просмотрите движение цен за полгода или год. Ищите уровни поддержки и сопротивления, а также линии тренда, которые подтвердили бы правильность ваших суждений.

С входом определились. Далее будем говорить о самом приятном: как фиксировать прибыль. Спасибо за внимание. Оставайтесь на блоге о трейдинге. Оставляйте комментарии. Удачных торгов!


Ссылка по теме: что дает владение акциями?

\subsection{Стратегия выхода свинг трейдера}

Привет, уважаемые читатели блога о трейдинге. Мы продолжаем изучать стратегию свинг трейдинга. И сегодня вы узнаете, как свинг трейдер фиксирует прибыль и контролирует убытки!

Стратегия выхода из сделки включает два аспекта:
\begin{itemize}
\item     Когда закрыть позицию, если цена двинулась против вас, и фиксировать убыток?
\item     Когда выйти из рынка, если акция пошла в вашу сторона, и фиксировать прибыль (об этом более подробно в следующей статье)?
\end{itemize}

На два этих вопроса есть два ответа, которые и составят вашу стратегию
выхода. Перед тем, как открыть позицию, определитесь, где и когда вы
будете выходить!

\paragraph{Стоп лосс ордер}

Есть два типа стопов: физический и ментальный. Первый – это реальный ордер, который вы выставляете сразу же после открытия позиции через своего брокера. Ментальный стоп лосс остается в вас на листе (или в уме) и когда цена достигает этого уровня, вы закрываете позицию маркет ордером. С технической точки зрения, не имеет значения, какой стоп ордер вы используете.

Стоп лосс – важный пункт вашего плана по риск менеджменту. Вы дисциплинированный трейдер и всегда следуете своему плану, не так ли? Не зависимо от того, какой тип стоп ордера вы используете, физический или ментальный, всегда выходите из рынка по плану!

Где должен стоять ваш стоп лосс? Во-первых, он должен отстоять не очень далеко, чтобы не сокращать размер позиции. Во-вторых, чтобы стоп ордер имел смысл, он должен стоять вне зоны рыночных шумов.

Посмотрите на средний диапазон цены за последние 10 дней, используя индикатор ATR. Если средний диапазон для выбранной вами бумаги равен, скажем, 1\$, то не имеет никакого смысла ставить стоп лосс менее этого значения от уровня входа, потому что вы покинете рынок преждевременно.

Если вы рассматриваете длинную позицию, ставьте стоп ордер либо чуть ниже минимума (2-5 центов) второго бара разворотной модели, либо за уровень поддержки. Пример:

На графике цена пробила важный уровень сопротивления по 12\$ и пошла
вверх. Восходящий тренд подтверждается перекрестом скользящих средних
(10 SMA выше 30 EMA). Поднявшись чуть выше 13\$, акция откатила в
активную зону трейдера, формируя разворотную модель в
основании. Предыдущая линия сопротивления становится поддержкой.

\paragraph{Стоп лосс, длинной в вечность}

Когда вы открываете позицию, то ожидаете движение цены в пределах нескольких следующих дней, верно? А что, если этого не происходит? Такое бывает, когда акция пребывает в очень узком боковом диапазоне в течении нескольких дней. Должны ли вы ожидать движения? Не стоит. Ваши деньги должны работать, поэтому поищите другие возможности (благо, их хватает).

Рассматривайте такие сделки, как бизнесмен служащих. Если ее действия не отвечают вашим требованиям – увольте ее.

Итак, ответ на первый вопрос мы получили. Приступим ко второму.

\paragraph{Закрытие позиции}

Используйте трейлинг-стоп. Это простой и лишенный эмоций путь выхода из прибыльной позиции. Если у вас открыта стандартная свинговая позиция на 2-5 дней, то переносите ваш стоп на несколько центов ниже минимума предыдущей свечи или текущей – в которой минимум ниже. Как здесь:

Если вы вошли на первом откате, то можете попробовать получить прибыль со всего движения. Такие трейды заметно могут пополнить ваш торговый счет. В таком случае переносите стоп под уровень каждого предыдущего отката, пока рынок сам не закроет вашу позицию. Вот пример:

В любом случае, вы должны определится, где должен стоять ваш стоп лосс и когда фиксировать прибыль. Используйте эту простую стратегию, которая сбережет ваши нервы и торговый капитал.

Оставайтесь с блогом о трейдинге. В следующей статье вы посмотрите, как эффективно использовать стоп лосс на примере. Спасибо за внимание. Оставляйте комментарии. Удачных торгов!


Интересная тема - акции наиболее крупных компаний США в индексе Доу

\subsection{Перенеси стоп лосс, сохрани прибыль!}

День добрый, читатели блога о трейдинге. В предыдущей статье "Стратегия выхода свинг трейдера" нами обсуждалась начальная позиция постановки стоп лосс. На этой странице мы рассмотрим, как переносить стоп лосс, если цена акции пошла в вашем направлении, чтобы сохранить и максимизировать прибыль. Ниже я предоставлю вам пример, в котором мы сделаем это шаг за шагом. Учтите: стоп лосс всегда будет выставляться после закрытия рынка.

Запомните следующие правило, которым мы будем руководствоваться, и которого нужно придерживаться, передвигая свой защитный ордер:

Переносите стоп лосс под минимум текущего дня или предыдущего, в зависимости, какой ниже.

Далее давайте рассмотрим пошаговый пример:

\paragraph{Начальный стоп лосс, день 1}

С графика видно, что акцию можно покупать на закрытии последней свечи, потому что:
\begin{itemize}
\item     Имеется восходящий тренд (10 SMA выше 30 EMA);
\item     Цена откатила в активную зону;
\item     Мы получили точку разворота в основании (желтый овал). Плюс к этому, в японском теханализе это важная свечная модель «утренняя звезда».
\end{itemize}

Заметьте, что откат состоит из шести свечей, причем не все они черные. Почему мы не покупаем раньше? Присмотритесь внимательно, ни одна свеча не закрывается выше максимума предыдущей свечи. Разворотная модель в основании формируется только тремя последними свечами (в желтом овале): минимум, за ним более низкий минимум и далее более высокий минимум, который закрывается на 5 центов выше максимума предыдущего дня (здесь нужно покупать).

Как мы уже знаем, начальный стоп лосс ставится за уровень разворотной
свечи, как показано на примере.

\paragraph{Перенос стоп лосс, день 2}

На следующий день после открытия позиции появилась белая свеча
(выделенная зеленым). Значит, цена пошла в нашу сторону и после
закрытия рынка мы хотим перенести стоп. Помним правило? Стоп лосс
перемещаем под минимум предыдущего дня или текущего, в зависимости,
какой ниже. Так как минимум предыдущей свечи ниже, значит, там мы и
поставим наш защитный ордер.

\paragraph{Перенос стоп лосс, дни 3-10}

В последующие несколько торговых сессий мы видим, что появляются новые более высокие минимумы и, соответственно, стоп лосс переносится выше, вслед за движением цены.

Запомните, если цена движется в вашу сторону, защитный ордер перемещается следом. Если цена начинает двигаться против вас, стоп лосс остается на прежнем месте. Никогда не передвигайте стоп лосс против своей позиции.

На 4 день мы вышли в безубыток (риск менеджмент: стратегия для начинающих), даже небольшой плюс. Теперь можно спокойно наблюдать за ростом прибыли.

Итак, позиция сохраняется на протяжении 10 дней. После закрытия
последней свечи наш стоп лосс перемещается под минимум предыдущей, так
как он более низкий. Заметьте, что 10 день открылся и закрылся
практически на одном уровне. В свечном анализе это разворотная модель
«звезда», которая может говорить об изменении тенденции.

\paragraph{Перенос стоп лосс, день 11}

11-ая свеча завершила формирование разворотноой свечной модели
«вечерняя звезда». Куда мы разместим наш защитный ордер? Так как
минимум предыдущего дня выше текущего, то стоп ставим под минимум
текущей свечи (на графике предпоследняя). И, как видите, следующая
торговая сессия закрывает нашу позицию, которая держалась 11 торговых
дней!

\paragraph{Некоторые советы}
\begin{itemize}
\item     Где ставить стоп лосс, если появилась свеча с большим диапазоном? Чтобы застраховать большую прибыль, перейдите на меньший таймфрейм (часовой, к примеру).
\item     Если цена подходит к уровню сопротивления, рассмотрите возможность закрытия половины позиции и размещение стоп лосс за этим уровнем (если бумага его удачно тестирует).
\item     Если вы зашли на первом откате и хотите «прокатится» до конца тренда, то эта стратегия вам не совсем подойдет. Переставляйте стопы под уровни предыдущих откатов.
\end{itemize}

Это довольно простой и надежный метод фиксирования прибыль. К тому же, он лишен догадок и эмоций. Просто передвигайте свой стоп лосс каждый раз после закрытия торговой сессии в свою сторону и никогда против позиции. Следующая наша тема — как согласовать нашу торговлю с остальным рынком. Блог о трейдинге благодарит за внимание. Оставляйте комментарии. Удачных торгов!


Стоит посмотреть - расширение Фибоначчи: как пользоваться?

\subsection{Как торговать в ногу с рынком?}

Добрый день, читатели блога о трейдинге. Если рынок движется вверх, то
три из четырех акций будут также двигаться вверх. И наоборот, если
рынок начинает нисходящую тенденцию, то три из четырех акций будет
двигаться вниз. Поэтому вы можете пожелать включить в свой торговый
план временную стратегию, которая помогла бы согласовать полученные
сигналы на открытие позиции с направлением движения рынка.

\paragraph{Временная стратегия с использованием скользящих средних}

Перед тем, как искать акции для торговли, сперва, откройте дневной график индекса S\&P500. Используйте тиккер либо SPY (рекомендую, это ETF, доступен индикатор объема торгов), либо SPX. Нанесите уже знакомые (как использовать скользящее среднее) 10 SMA и 30 EMA, которые помогут вам определится с направлением торговли. Вот правило:
\begin{itemize}
\item     Если 10 SMA над 30 EMA, рассматривайте акции лишь на длинную позицию;
\item     Если 10 SMA под 30 EMA, рассматривайте акции лишь на короткую позицию.
\end{itemize}

Эта простая техника поможет вам оставаться в правильном русле.

Посмотрите, как это выглядит на примере:

На графике вы видите, как легко и быстро можно определится с направлением позиции. Но запомните одну вещь:

Скользящее среднее – это трендовый индикатор. Поэтому, когда рынок входит в горизонтальный коридор, он подает много ложных сигналов.

Хорошо. Теперь мы знаем, когда нужно рассматривать длинную позицию, а когда короткую. Далее давайте узнаем, а когда, собственно, покупать и, когда продавать. Для этого будем использовать осциллятор Williams \%R.

\paragraph{Временная стратегия с использованием Williams \%R}

Я не являюсь большим поклонником разных индикаторов, но процентный диапазон Вильямса показал себя полезным в распознавании краткосрочных максимумов и минимумов, а также возможных их разворотов. Он рассчитывает цену закрытия по отношению к диапазону за заданное число дней.

На большинстве торговых платформ этот индикатор имеет предустановленный период 14. Но нам нужно, чтобы он был более чувствителен. Поэтому мы будем использовать 3-дневный период. Вот правило для временной стратегии с использованием Williams \%R:
\begin{itemize}
\item     Когда 10 SMA над 30 EMA, рассматриваем длинную позицию, если Williams \%R ниже уровня -80 (перепродан);
\item     Когда 10 SMA под 30 EMA, рассматриваем короткую позицию, если Williams \%R выше уровня -20 (перекуплен).
\end{itemize}

Чем более перекуплен или перепродан рынок, тем выше возможность разворота. Ищите ситуации, когда Williams \%R ниже уровня -90 для лонга и выше -10 для шорта.

С графика видно, как мы игнорируем возможности для открытия коротких позиций, когда процентный диапазон Вильямса показывает перекупленность, если 10-периодный SMA выше 30-периодного EMA. Тоже касается обратной ситуации. Старайтесь торговать в сторону действующего тренда.

С помощью индикаторов мы не предугадываем будущее поведение цен. Это
пустая трата времени. Реагируйте на сигналы, которые рынок подает в
данный момент.

\paragraph{Импульсный режим}

Время от времени, рынок акций входит в, так называемый, импульсный режим. В этом случае Williams \%R не подает сигналов. Когда такое происходит, всего несколько акций могут дать откаты, которые мы бы смогли торговать.

На графике SPY процентный диапазон Вильямса практически 2 месяца не выходит за уровень перепроданности (-80), не подавая сигналов на покупку. Если в это время вы попробуете поискать акции с необходимыми нам данными, то их окажется очень немного.

Если рынок пребывает в импульсном режиме, то у вас есть два выхода:
\begin{itemize}
\item     Не делать ничего
\item     Перейти на меньший таймфрейм, скажем, 1h.
\end{itemize}

На часовом графике снова появились сигналы на покупку. Значения
индикатора Вильямса ниже -80

\paragraph{Заключение}

Используйте описанную временную стратегию, чтобы торговать акциями в соответствие с обще рыночным движением. Но она не является на 100\% обязательной (с этим вы разберетесь по ходу личной практике).

Запомните следующие советы:
\begin{itemize}
\item     Когда у вас открыта позиция, забудьте о индикаторах и всем остальном. Руководствуйтесь лишь знаниями, полученными в статье «стратегия выхода».
\item     Все индикаторы, включая скользящее среднее и Williams \%R, являются субъективными. Цена нет. Поэтому имейте больше доверия к различным уровням и линиям тренда.
\end{itemize}

В следующей статье я изложу всю стратегию свинг трейдинга в схематичном виде. Блог о трейдинге благодарит за внимание. Удачных торгов!


Это познавательно: чем характеризуется медвежий тренд?

\subsection{Схема стратегии свинг трейдера}

Приветствую вас, читатели блога о трейдинге. На протяжении шести
постов мы разбирали стратегию свинг трейдинга. Здесь вы найдете
краткую ее схему. Весь объем предыдущего материала здесь изложен
кратко, в несколько строк. Можете распечатать данную схему и положить
возле своего монитора для удобства, как подсказку.

\subsubsection{Временная стратегия}

Используйте график SPY с нанесенным на него скользящими средними и Williams \%R, чтобы отслеживать движения всего рынка. Открывайте позиции, только когда будут выполнены следующие условия:

Скользящие средние: если 10 SMA над 30 EMA – восходящая тенденция; если 10 SMA под 30 EMA – нисходящая.

Williams \%R: настройте период за три дня; если значение ниже -80 – рынок перепродан; если значение выше -20 – перекуплен.

Длинная позиция: 10 SMA над 30 EMA и по Williams \%R рынок перепродан.

Короткая позиция: 10 SMA под 30 EMA и по Williams \%R рынок перекуплен.

Меньший таймфрейм: используйте 1h график SPY для подтверждения бычьих или медвежьих свечных моделей.

Примечание: пользуйтесь этой стратегией только для выбора удачного
момента входа на рынок, но не для выхода. Когда есть открытая позиция
по какой-либо акции, управляйте правильно стоп лоссом и рынок сам
закроет вашу позицию.

\subsubsection{Отбор акций и стратегия входа}

Это схема для длинной позиции. Для короткой наоборот.

Тренд: должен быть восходящим – 10 SMA над 30 EMA; бумага торгуется между 10 SMA и 30 EMA в активной зоне (АЗТ); акция пребывает во второй фазе рынка.

200 SMA: бумага торгуется над ней.

Откат: должен состоять по крайней мере из 3 свечей.

Свечная модель: сформировалась либо разворотная точка в основании, либо бычья разворотная модель свечного анализа.

Меньший таймфрейм: ищите подтверждение на часовом графике.

Earnings: не должны оглашаться в течении нескольких дней по выбранной акции.

Уровень поддержки: бумага торгуется от уровня поддержки.

\subsubsection{Стратегия выхода и риск менеджмент}

Стоп лосс: для ограничения убытков используйте ментальный или физический стоп лосс за уровень поддержки или минимума разворотной модели, но не меньший, чем средний истинный диапазон (1ATR).

Если цена идет в вашу сторону, переносите стоп лосс под минимум предыдущей или текущей свечи, какой ниже, пока рынок сам не закроет вашу позицию.

Риск менеджмент: используйте правила данные в статье "Основы риск
менеджмента для свинг трейдера". Если вы начинающий трейдер
рассмотрите вот эту стратегию.

\subsubsection{Некоторые советы}

Первый откат: появляется после, например, пробоя важного уровня сопротивления или линии тренда. Может стать для вас лучшей торговой возможностью.

Движение цены: степень движения цены в предыдущий день, определяет возможный диапазон движения текущего дня.

Пример:

Это все. Работайте, практикуйтесь, вносите какие-то свои конструктивные изменения. Самое главное, чтобы вы поняли суть стратегии свинг трейдинга: покупать акцию нужно после ее снижения, возле уровня поддержки. С продажей без покрытия наоборот. Всегда избегайте бумаги с близкой датой выхода отчета о прибылях (EPS). В этот день акция может торговаться непредвиденно. В статье "Быстрый фильтр акций под нашу стратегию свинг трейдинга" я покажу, как быстро отбирать бумаги для торгов. Блог о трейдинге благодарит за внимание. Будьте успешными!


Читайте также на блоге: анализ ликвидности акций

\subsection{Что такое гэп и его анализ?}

Доброго времени суток, читатели блога о трейдинге. Гэп, или ценовой разрыв, или окно (в свечном анализе), как вам нравится, в зависимости от того, где он возникают на тренде, несет разную информацию для свинг трейдера. Он можетт указать вам на начало нового мощного движения в цене акции, на то, продолжиться тренд или развернется. Такая высокая информативность, что несет гэп, делает его ценным графическим элементом.

На этой странице мы разберем, что такое гэп, как анализировать гэп, и
самое главное, я вам расскажу, какие гэпы формируются
профессиональными трейдерами, а какие начинающими.

\subsubsection{Что такое гэп?}

Это ценовой разрыв на графике, где не происходили торги. Он может появляться на любом таймфрейме, но нам, как свинг трейдерам, наиболее предпочтителен дневной.

Гэп на дневном графике происходит, если бумага открывается выше максимума (или ниже минимума) предыдущего дня. Почему это происходит? Потому, что продавцы или покупатели массово выставляют ордера на покупку или продажу перед открытием рынка. Как правило, это происходит, когда выходят важные новости. Рассмотрим пример:

Компания AOL перед выходом отчета о доходах (earnings) имела максимум
дня на уровне 27.96\$. Их отчет превысил ожидание и вызвал волнение в
инвесторов. Ордера на покупку начали выставляться до открытия торговой
сессии. Следующий день AOL открыла на уровне 28.19\$ и ушла
вверх. Получился ценовой пробел в 23 пункта, где отсутствовали
сделки. Это и есть гэп.

\paragraph{Заполнение гэпа}

Иногда мы слышим фразу, типа «заполнение гэпа» или «гэп заполнился». О чем идет речь?

Когда акция торгуется в промежутке предыдущего ценового разрыва, мы говорим, что гэп заполнился. Посмотрите пример:

На протяжении более года ценовой разрыв служил уровнем сопротивления, пока не был прорван. Далее гэп стал стремительно заполняться.

В свечном анализе ценовые разрывы принято называть окнами. Когда гэп заполняется, японцы говорят «окно закрывается».

Некоторые трейдеры говорят, что гэпы всегда заполняются. Другие это
отрицают. Я не думаю, что эта тема стоит спора. Хочу только сказать:
есть ценовые разрывы, которые заполняются в течении продолжительного
времени, а также те, которые не заполняются на протяжении годов. Может
последним требуется больше времени? Это не столь важно.

\paragraph{Виды гэпов}

В зависимости от того, где ценовые разрывы появляются на графике, различают 3 их типа:
\begin{itemize}
\item     Гэп на разрыв – возникает, когда цена выходит из зоны консолидации или, как завершение какой-то графической модели. Формируются профессиональными трейдерами.
\item     Гэп на отрыв – возникает после сильного движения цены, как правило, между свечами с большими диапазонами и говорит о силе тренда. Еще называется измерительным. Указывает на продолжение тенденции.
\item     Гэп на излет – возникает в направлении главной тенденции и
  говорит о финальной волне покупок/продаж перед изменением
  тренда. Формируются начинающими трейдерами.
\end{itemize}

\paragraph{Анализ гэпа: профессиональные трейдеры и начинающие}

Мы подошли к основной теме обсуждения. Когда появляется ценовой разрыв, важно понимать, кто его сформировал. Если за этим стоят профессиональные трейдеры, то вы можете быть уверены, что тренд продолжиться. Если его сформировали начинающие трейдеры, то скорее всего, тренд приближается к развороту.

Для начала, запомните одну вещь. Профессионалы покупают после того, как была совершенна волна продаж и продают, когда прошла волна покупок. То есть, это начальная фаза тенденции после разворота, или пробоя уровня.

Начинающие трейдеры делают все в точности наоборот. Когда цена уже на протяжении нескольких дней движется вверх, они боясь, что упустили хорошую возможность, начинают покупать. В это время профессиональные трейдера уже продают.

Вот пример гэпа, образованного начинающими трейдерами:

Как видите, здесь главенствуют эмоции – не лучшие друзья трейдера. Покупки совершаются, после нескольких дней роста подряд в конце восходящей тенденции.

А вот пример в обратную сторону:

Второй ценовой разрыв возник вначале тенденции и за ним сразу последовал стремительный рост цены акции.

Теперь выведем правила:
\begin{itemize}
\item     Когда вы видите гэп, появившийся вначале тенденции, после волны продаж, либо при пробои предыдущего экстремума, то ждите стремительного роста цены
\item     Если гэп возникает после продолжительного тренда, то это сигнализирует его конец.
\end{itemize}

Гэп является отличными индикатором направления движения цены. Он дает нам великолепные возможности, как свинг трейдерам: указывают на зарождение трендов и на достижение экстремумов. А в основе лежит психология разных трейдеров. Блог о трейдинге благодарит за внимание. Оставляйте комментарии по теме гэп и его анализ. Торгуйте с умом!


Прочтите также: выбор брокера на фондовом рынке

\subsection{Как с помощью индекса волатильности VIX можно определять смену рыночного тренда в около 70\% случаев?}

Добрый день, читатели блога о трейдинге. Индекс волатильности VIX, или индекс волатильности может использоваться вами для согласования открытия позиции с рынком. Это еще одна временная стратегия, разработанная Ларри Коннорсом, и стала известной, как Connors VIX Reversals (развороты индекс волатильности VIX Коннорса). Она используется для определения момента, когда весь рынок (S\&P 500), скорее всего, развернется в обратном направлении.

Всегда помните об этом индикаторе и пользуйтесь им в дополнение к
вашей постоянной временной стратегии.

\subsubsection{Что такое индекс волатильности VIX?}

Индекс волатильности  (VIX) измеряет будущую волатильность. Он обеспечивает нас отличной индикацией уровня страха и жадности на рынке.

Волатильность всегда стремится возвратиться к своим средним величинам. Это значит, что периоды с высокой волатильностью, в конечном счете, опустятся к своему среднему значению. И наоборот, периоды с низкой волатильностью, рано или поздно, поднимутся к средним значениям.

Высокие цифры обычно свидетельствуют о перепроданости рынка, и нам необходимо искать возможности для открытия длинных позиций. Низкие цифры обычно возникают при перекуплености рынка, и мы фокусируемся на коротких продажах. Мы всегда работаем против рыночной толпы!

Смотрите пример за 2-хлетней период, как вершины на VIX совпадают с впадинами на SPY:

Существует, по крайней мере, 10 различных типов разворотов VIX (так
называемые сигналы CVR). Мы растворим два из них:

\paragraph{Использование 10-периодного скользящего среднего}

Первый тип, из рассматриваемых нами CVR сигналов, предлагает применять 10-периодное простое скользящее среднее (10 SMA на графике). Когда VIX превышает скользящее среднее на 10\%, на рынке (S\&P 500) будет перепроданность. То есть, он достиг своего минимума и вероятно развернется вверх.

Правильность такого сигнала отмечается в около 70\% случаев! Поэтому следите за ним.

То же верно и для установок на короткие продажи. Ждите, когда VIX
уйдет ниже 10-периодной скользящей средней на 10\%, для открытия коротких позиций.

\paragraph{Использование RSI индикатора}

Другой вариант – это применение индикатора RSI с установленным периодом 5 свечей. На графике выше RSI находится в нижней его части.

Когда RSI поднимается выше 70, то VIX является перекупленным, а рынок перепроданным. В таких случаях нужно искать возможности для открытия длинных позиций.

Когда RSI опускается ниже 30, то VIX является перепроданным, а рынок перекупленным. Значит, нужно остановиться на продажах.

На графике зоны перепроданности и перекупленности RSI отмечены зеленым и красным цветами соответственно.

Запомните очень важную вещь: развороты VIX используются для идентификации экстремумов рынка S\&P 500. Таким образом, чтобы эти сигналы имели значения, вам нужно торговать либо самим индексом S\&P 500 (SPY), или найти графики акций, которые двигаются подобно ему.

Если вас заинтересовали развороты VIX, то можете почитать книгу «Short Term Trading Strategies That Work». Там найдете еще несколько неплохих стратегий.

Как только вы начнете изучать развороты индекс волатильности VIX и видеть, как часто рынок действительно разворачивается, когда несколько разных сигналов указывают в одном направлении, то поймете, насколько ценным это знание является. Вы получите большое преимущество над другими трейдерами! Блог о трейдинге благодарит за внимание. Торгуйте в ногу с рынком!


Интересно: что такое маржа простым языком?

\subsection{Как выбрать акции, которые обыгрывают рынок S\&P 500?}

Доброго времени суток, читатели блога о трейдинге. Для того, чтобы отобрать акции, которые обыгрывают рынок, нам нужно определить их относительную силу. Относительная сила показывает, насколько движение цены (или тренд) какой-то акции сильнее или слабее другой. Относительную силу можно также применять для сравнения со всем рынком или с определенной индустрией.

Если акция имеет тренд сильнее, то мы говорим, что она имеет относительную силу. Если слабее, то относительную слабость.

Для того, чтобы улучшить ваши шансы на успешную сделку, вам необходимо покупать только те бумаги, которые сильнее всего рынка, а продавать те, которые слабее. Единственный путь достичь этого – сравнить выбранный график с индексом S\&P 500.

S\&P 500 можно наложить прямо на график вот таким образом:

На изображении выше, указан свечной график COP, а непрерывной линией – индекс S\&P 500.

От начала видимого графика COP до декабря 2013 видно, что акция имеет тренд сильнее рыночного. Таким образом, мы говорим, что она сильнее относительно рынка.

Далее свечи пересекают сверху вниз линию индекса и акция становиться
явно слабее S\&P 500. В таких случаях, мы говорим, что она слабее
относительно рынка.

\subsubsection{Применение относительной силы к индустриям}

Вы также можете сравнивать акцию со всей индустрией, в которой она находятся. Открывая длинную позицию, вам необходимо покупать бумаги, которые сильнее относительно их индустрий. Когда предполагается короткая позиция, нужно продавать акции, которые слабее относительно их индустрий.

Для получения еще большего преимущества можно сравнивать разные индустрии между собой. Например, при длинной позиции, вы ищите акции в индустрии, которая сильнее относительно других на рынке.

Хорошо, думаю все согласны, что необходимо покупать бумаги относительно сильные, а продавать относительно слабые. Осталось только найти таких кандидатов и торговать ими. Но подождите, если у меня нет возможности наложить S\&P 500 на график? Есть ли простой путь их поиска? Да!

\subsubsection{Поиск относительной силы или слабости}

Дождитесь, когда рынок хорошо просядет на протяжении нескольких дней. Теперь пройдитесь по акциям в вашем списке. Которые из них еще росли в день, когда рынок уже распродавался? Эти бумаги и будут иметь относительную силу, и их вам нужно покупать, когда наступит момент. Почему? Потому что эти акции будут расти одними из первых, когда рынок развернется и показывать наибольшие результаты.

Также смотрите на графики различных индустрий. Которые из них росли во время большого медвежьего дня на рынке? Эти индустрии и будут иметь относительную силу. Акции, которые в них входят, нужно покупать.

Когда планируются короткие продажи, делаем все тоже с точностью наоборот.  Когда рынок оканчивает день большой бычьей свечой, нужно искать индустрии и акции, которые продолжали падать. Они слабее относительно рынка, и их нужно продавать.

Не нужно недооценивать эту простую технику! Если вы в основном нацелены на покупку, то используйте дни распродаж на рынке, для поиска акций, которые продолжают расти. Далее подождите нужного момента. Они, вероятно, выстрелят раньше и сильнее остального рынка.

Изучая и применяя такие маленькие торговые секреты, вы всегда будете выше остальной публики! Относительная сила должна использоваться для вашей выгоды. Далее мы поговорим о самой важной главе вашего торгового плана — риск менеджменте. Блог о трейдинге благодарит за внимание. Будьте успешными!


Полезная информация: индикатор показывающий направление тренда

\subsection{Основы риск менеджмента для свинг трейдера}

Приветствую вас, читатели блога о трейдинге. Уильям О'Нилл как то сказал: «Весь секрет успеха на фондовом рынке состоит в том, чтобы терять как можно меньше денег, когда вы неправы». Я полностью с этим согласен. Рынок не поддается контролю. Зато контролировать можно деньги и риски. Этим вы должны заниматься всякий раз, когда открываете сделку. Это  составляет главнейшую главу вашей торговой стратегии, которая называется риск менеджмент.

Риск менеджмент состоит с трех частей: управление капиталом, позицией
и эмоциями. Несколько следующих правил помогут вам в управлении этими
тремя составляющими риск менеджмента.

\subsubsection{1. Риск менеджмент: правило двух процентов}

Касается, непосредственно, вашей главы по управлению капиталом. Суть его – это рисковать в каждой сделке не более 2\% от вашего торгового капитала. Если акция не соответствует этому критерию, то позиция не открывается.

Пример:

    Ваш торговый капитал составляет 5000\$; 2\% – это 100\$;
    Значит, вы рассчитываете размер позиции в каждой сделки таким образом, чтобы рисковать не более 100\$. Иными словами, если акция имеет сильный уровень поддержки от \$20 (отвечает уровню покупки) до \$19.50 (установка стоп лосс),  размер вашей позиции равняется 200 акций [100/(20-19.50)]

На самом деле это очень популярная и разрекламированная
формула. Конечно, вы можете иметь свою. Но, на мой взгляд, правило 2\% очень просто позволяет контролировать свои финансы. Если ему следовать, то это позволит вам избежать таких курьезных ситуаций, когда буквально за несколько торговых сделок ваш брокер присылает вам письмо, в котором просит пополнить счет.

\subsubsection{2. Отношение потенциала прибыли к риску}

Это правило посвящено управлению капиталом целиком. Если вы видите акцию, которая не соответствует ему, то позиция не должна открываться вообще.

Бывают ситуации в свинг трейдинге, когда откат на тренде происходит от вершины с сильным сопротивлением. Такие откаты мы называем нерентабельными, поскольку нет запаса хода для цены акции. Об этом на примере рассказывается в статье "10 советов по ценовым движениям, которые улучшат вашу торговлю".

Когда вы торгуете на рынке ценных бумаг, уровень стоп лосс соответствует риску, а точка предполагаемого выхода (устанавливается в зависимости от стратегии) – прибыли. Это соотношение должно быть не менее 3:1. Если это правило соблюдать, то даже 30-40\% позитивных результатов будут давать доход.

Пример:

    Трейдер делает 3 сделки со стоп лосс равным 200\$, а соотношение прибыль/риск 3:1;
    Первая дает результат «-200\$»;
    Вторая – «+600\$»;
    Третья – «-200\$»;
    Суммарный результат составляет 200\$.

А если соотношение будет, скажем, 5:1, то прибыль составит 600\$. И так далее. То есть, в нашем случае, даже если вы делаете 33\% позитивных сделок, то, в конечном счете, результат будет для вас позитивным.

\subsubsection{3. Следуйте беспрекословно своей торговой стратегии}

анная глава риск менеджмента относится к психологии трейдинга, то есть управлению эмоциями. Больше информации на странице "Психология трейдинга и дисциплина: как торгуют другие и как их переиграть".

Никогда не открывайте позицию, если она не соответствует вашим правилам. Будьте дисциплинированными. Независимо от того, долгосрочный ли вы трейдер и опираетесь на фундаментальные данные, или краткосрочный и торгуете какие-то определенные графические модели, всегда держитесь в рамках своей торговой системы.

Одна из самых частых ошибок (часто можно увидеть в трейдерских чатах)
это торговать акции по чьи-либо совету, не делая анализа
самостоятельно. Итогом этого станет «похудение» торгового счета.

\subsubsection{4. Риск менеджмент: дневник трейдера}

Это правило относится к управлению эмоциями (как не странно) и я рекомендую додерживаться его каждому. Записывайте все сделки в отдельный блокнот. У вас может возникнуть вопрос: «Зачем вести дневник трейдера, если большинство брокеров предоставляет информацию о торговом балансе и отдельных позициях?». Делом в том, что:

    Нужно дать ответ на вопрос «Почему?». Почему я открыл данную позицию? Почему поставил стоп лосс на этом уровне? И так далее. То есть, объясните себе самому каждый свой трейд. Потом, в конце учетного периода, проведите анализ. Вы увидите, насколько легко будет замечать и исправлять ошибки.
    Когда записываешь все данные самостоятельно, как бы вливаешься
    или, даже сказал бы, сродняешься с работой на фондовом рынке.

\subsubsection{5. Риск менеджмент: стоп лосс}

Это последнее, но очень важное, правило свинг трейдинга, относящееся к управлению позицией. Оно очень простое, но часто игнорируется. Запомните, открытая позиция должна иметь стоп лосс. Любимое выражение успешных трейдеров: «Ограничьте свои потери, а прибыли позаботятся о себе сами!».

Этих пять простых правил риск менеджмента, при должном их соблюдении, могут изменит вашу торговлю акциями раз и навсегда. Заметьте, что на долю управления капиталом и управления эмоциями перепадает 80\% (по 40\% каждый). Их роль настолько важна в риск менеджменте, потому что мы не можем контролировать рынок и быть всегда правыми. Но мы можем контролировать себя и свои деньги. Блог о трейдинге благодарит за внимание. Успешных торгов!


Статья по теме: как определить флет или тренд?

\subsection{Риск менеджмент: стратегия для начинающих}

Здравствуйте, читатели блога о трейдинге. Риск менеджмент для начинающих должен соблюдать главную задачу — сохранение торгового счета. Стратегия два к одному – это очень осторожный способ торговли, однако, если вы в трейдинге новенький, то это хороший шанс для вас остаться на плаву, пока не начнете регулярно побеждать рынок. Этот подход к риск менеджменту поможет вам минимизировать убытки, оставаясь с неплохой прибылью. Как говорится: «Вы не можете обанкротиться, получая прибыль».

Основная мысль стратегии два к одному – закрывать половину позиции,
соответственно, забирать половину прибыли, если цена прошла в вашу
сторону на величину равную стоп лоссу.

\subsubsection{Пример риск менеджмента 2 к 1}

Рассмотрим свежий пример с акцией ITMN.

Мы, как свинг трейдеры, ищем акции с определенными характеристиками. В желтом овале вы видите стандартную разворотную модель в основании. На уровне 9,7\$ (зеленая линия) была совершенна покупка. Почему я покупал:
\begin{itemize}
\item     Сформировался восходящий треугольник с уровнем сопротивления 10,0;
\item     Скользящие средние после хаотичного движения показывает начало восходящего тренда (10 SMA над 30 EMA);
\item     Акция дала откат в активную зону (АЗТ);
\item     Выделенные желтым три свечи сформировали точку разворота в основании. К тому же, две последние из них – бычья модель поглощения в свечном анализе.
\end{itemize}

Стоп лосс ставим под минимум средней свечи разворотной модели на уровне 9,4. Итак, наш риск составляет 30 центов или 0,3\$.

Следуя нашей стратегии, когда цена поднимется к 10.0, то есть прирост равен величине стопа, мы зафиксируем половину прибыли. Что нам это дает? Ваша сделка теперь при любых условиях будет положительной. Ну, если не считать комиссионные.

Вы заметили, что я не покупал на пробое уровня 10,0? Это более рискованно, причем неделю назад бумага уже тестировала этот уровень – неудачно. Акция дала возможность войти раньше. Если следовать стратегии 2 к 1, и закрыть половину позиции, то даже при очередном неудачном пробое уровня 10.0, мы ничего не теряем.

Как видите, следующий день закрылся для нас с прибылью. Мы спокойно теперь наращиваем доход, управляем нашим стоп лоссом и ждем, пока рынок сам не закроет нашу позицию, которая, как уже ясно, будет в плюс.

Стратегия два к одному – неплохой риск менеджмент для начинающих. Это сберегательная тактика для вашего капитала. Когда открытие и удержание позиции на протяжении нескольких дней или даже недель станет для вас обычным делом, тогда переходите на более агрессивные стратегии риск менеджмента. В следующей статье я покажу вам, как сделать риск одинаковым на разных рынках, с любой волатильностью. Блог о трейдинге благодарит за внимание. Оставляйте комментарии. Удачных торгов!


Возможно вас заинтересует: брокеры фондового рынка с демо счетом

\subsection{Размер позиции – укрощение волатильности}

    «Я говорил, что мы несли небольшие убытки, но думаю, что лучше нести оптимальные убытки. Вы не хотите проиграть слишком много, но также не нужно, чтобы стопы были настолько короткими, чтобы легко вылететь в трубу.»

    Джерри Паркер

Приветствую вас, читатели блога о трейдинге. Размер позиции определяет количество акций/контрактов/валютных пар, торгуемых в соответствие с размером счета. Этот пункт рассматривается одним из первых при составлении стратегии риск менеджмента.

Вы уже решили, как будете определять размер позиции в каждой конкретной сделке? Нет. Не страшно, потому что в этой статье мы рассмотрим одну из лучших стратегий, которую использовали знаменитые «трейдеры-черепахи».

Несмотря на то, что размер позиции является жизненной базой для трейдинга, большинство трейдеров торгуют либо стандартным количеством лотов, либо, по своему усмотрению – «на глаз».

Мы же рассмотрим стратегию, которая рассчитывает размер позиции на основании волатильности. Другими словами, на каком бы рынке вы не торговали, ваши риски будут уравновешены. Итак, давайте приступим!

Управление рисками начинается с измерения волатильности рынков. Для ее расчета мы будем использовать истинный диапазон (ИД) – это разница между максимумом и минимумом цен бара или свечи. Затем определим средний истинный диапазон (СИД) – это скользящее среднее значение истинных диапазонов за период времени. В результате мы получим примерную волатильность.

Вы можете легко определять СИД самостоятельно. Для этого возьмите максимальные и минимальные значения цен последних 15 свечей и рассчитайте ИД для каждого. Затем суммируйте полученные данные (15 истинных диапазонов) и разделите на 15. Повторяйте эту процедуру с появлением каждой последующей свечи, но не забывайте отбрасывать значение диапазона самой последней свечи.

Вам может показаться этот процесс сложным и емким, рассчитывая волатильность таким образом для каждой бумаги. Но многие компьютерные программы делают это автоматически. Например, я пользуюсь Thinkorswim и здесь присутствует такой полезный индикатор, как AverageTrueRange (ATR), который и соответствует СИД (ATR=СИД=волатильность). В настройках я выставляю период 20, что дает мне возможность следить за волатильностью с периодом в 20 свечей.

Вы можете спросить, зачем нужно измерять волатильность? Ответ в том, что уровень стоп лосс у нас теперь будет приравниваться к ATR. Например, «черепахи» устанавливали стопы равные 1ATR или 2ATR. Здесь есть большие преимущества:
\begin{itemize}
\item     Не нужно гадать, где ставить ограничительный стоп;
\item     Ваши риски оптимальны на любом рынке;
\item     Легко можно экспериментировать.
\end{itemize}

Теперь мы знаем волатильность рынка. Определимся с размером риска в каждой сделке. В предыдущей статье "Основы риск менеджмента для свинг трейдера" упоминалось такое важное правило, как: «не стоит рисковать более 2\% своего торгового счета в каждом отдельном трейде».

Используя все полученные знания, выводим формулу для расчета размера позиции:

Размер позиции = 2\% * торговый счет/ATR

Значения «2\%» и «ATR» могут изменяться. Это зависит от выбранной вами торговой стратегии.

Пример:

Акция CKP пробила сильный уровень сопротивления (уровень 10\$ и уровень гэпа) и, что важно, закрывает последнюю торговую сессию над ним. Мы хотим купить эту бумагу лимит ордером на откате цены по 10.05\$.

Наш торговый счет составляет 5000\$. Стоп лосс будет выставляться на уровне 2ATR (на отметке 9.53\$). Рискуем 1.5\% нашего торгового счета. Все данные есть. Подставляем их в формулу и высчитываем размер позиции.

(1,5\% * 5000/100\%)/(2 * 0,258)= 145,3

Размер нашей позиции при условии срабатывания лимит ордера будет 145 акций, а если вы торгуете лотами, то уменьшаем до одного лота.

Еще раз замечу, что эту стратегию использовали «трейдеры-черепахи». И запомните, чем дальше вы ставите стоп лосс, тем меньше размер вашей позиции. Используйте оптимальные стопы. Блог о трейдинге благодарит за внимание. Оставляйте свои комментарии. Удачных торгов!


Следующая статья по трейдингу: индексы фондового рынка России

\subsection{Стоит ли вести дневник трейдера?}

Здравствуйте, читатели блога о трейдинге. Многие трейдера ведут дневник трейдера, куда детально вписывают всю на их взгляд необходимую информацию по сделкам. Еще больше трейдеров не ведут никаких дневников, так как, по их словам, на сайте брокера есть вся необходимая информация по их торговле. И хотя дневник трейдера не является обязательным атрибутом успешного торговца акциями, он имеет много плюсов, на которые следует обратить внимание.

Итак, давайте в этом посте рассмотрим: для чего ведутся дневники
трейдера, какую информацию они должны содержать, как их вести и,
наконец, стоит ли вообще их вести.

\subsubsection{Что содержит дневник трейдера}

Каждый специалист имеет при себе какую-то документацию для заполнения, по которой в дальнейшем ведется статистика. В принципе, с той же целью трейдер ведет дневник. Чтобы знать, как работает торговая стратегия, нужно проводить ее анализ. А для этого нужны данные прошлых сделок. Что это за данные:
\begin{itemize}
\item     Дата открытия и закрытия сделки
\item     Время открытия и закрытия
\item     Тиккер акции
\item     Направление сделки (buy или sell)
\item     Цены открытия и закрытия
\item     Количество проторгованных акций
\item     Способ входа в позицию (тип ордера)
\item     Результат
\end{itemize}

Выше представлены базовые значения, которые вы можете при
необходимости дополнять информацией, которая важна для вашей торговой
стратегии. Например, можно расширить информацию о компании (сектор,
индустрия, экономический рейтинг), или добавить размер комиссионных. В
столбике результат пишите сумму заработка или потерь в долларах или в
пунктах.

\paragraph{Дополнительная информация}

Статистику по тем нескольким пунктам, что были указаны выше, вы можете без проблем посмотреть на сайте своего брокера. В связи с этим, многие трейдеры не ведут дневников, чтобы лишний раз не дублировать информацию.

Впишите еще следующие пункты в свой дневник трейдера:
\begin{itemize}
\item     Уровень стоп лосса
\item     Оценка сделки
\item     Комментарии
\end{itemize}

Теперь кратко по каждому. Не знаю, есть ли такой брокер, который бы в статистику включал уровень стоп лосса. Так как это очень важный пункт, поскольку характеризирует эффективность и соблюдение вашей стратегии риск менеджмента, его нужно обязательно учитывать в своем дневнике.

\subparagraph{Что понимать под «оценкой сделки»?} Значения: в этом столбике ставите цифру от 1 до 5. Назначение: оцениваете, насколько правильно и точно вы следуете своей торговой системе. Применение: немножко поторговав, вы уже будете понимать, где недопустимая ошибка, грубая, сносная и легкая. Я, например, на листе бумаги написал напротив каждой оценки условия ее выставления. Это делает оценивание более объективным.

\subparagraph{Что нужно комментировать? }Сделали ошибку, нарушили условия своей торговой стратегии, запишите: что, как, почему. Другими словами, проведите анализ. Другое применение комментариям – это дать свое суждение ситуации на рынке. Я, например, каждое воскресение смотрю график SPY, затем сектора. Определяю для себя по текущей ситуации направление торгов, а также акции из самых сильных отраслей.

\subparagraph{Выводы:} чтобы заполнить последние три пункта, времени нужно больше, чем для заполнения всего дневника трейдера. Но! В одной из статей мы говорили об эмоциях, психологи трейдинга и дисциплине. Мое мнение, весь контроль эмоций во время торгов обеспечивается с помощью пунктов «оценка сделки» и «комментарии». Дальше решать вам.

\paragraph{Как вести дневник трейдера}

Когда я только начинал торговать, у меня была (и сейчас где-то есть) таблица в Excel. Там практически все, что указано выше присутствовало. Плюс разные полезные приложения, в основном касающиеся статистики (диаграммы, графики, таблицы).

Но, очень скоро я перешел на лист бумаги и карандаш. Всем давно известно, что такая практика более эффективна со стороны психологии. Так что у вас есть выбор между дневником трейдера в виде электронных таблиц, специальных программ и простого листа бумаги с карандашом.

Несмотря на то, что дневник трейдера не является проводником к успеху, вы увидели, что он имеет массу преимуществ. Я бы даже сказал, что его единственным недостатком является трата времени. Но, как по мне, это нужная трата. Блог о трейдинге - это также своеобразный дневник трейдера. Спасибо за внимание. Контролируйте свои риски и эмоции!



Интересная информация - фигура технического анализа восходящий клин

\subsection{Ответ на вопрос читателя: сколько денег нужно для старта в
  трейдинге?}

С Рождеством вас, читатели блога о трейдинге. Один из посетителей сайта задал вопрос: «Сколько денег нужно для торговли среднесрок?». Поскольку у меня сразу возникло множество мыслей, то решил написать в ответ целую статью. Вопрос данного типа относится к риск менеджменту, над темой которого я сейчас работаю. Конечно, посты на эту тему есть, но они не систематизированы. Я, как вы, наверное, заметили, стараюсь сделать каждую следующую статью дополнением предыдущей, чтобы раскрыть тему наиболее широко и связать весь материал в логическую цепочку.

Короче говоря, пускай эта статья станет первой в серии постов
посвященных риск менеджменту. И рассматривать мы будем здесь некоторые
рекомендации по размеру торгового счета, мой личный опыт и, вообще,
главное ли это и нужно ли обращать на него внимание.

\subsubsection{Какой размер торгового капитала рекомендуют
  специалисты?}

Книга Роберта Дила «Стратегия электронного дейтрейдинга» стала одной из первых, которую я прочел на тему биржевой торговли. В ней Дил дает советы по размеру торговых счетов для разных стилей трейдинга. Давайте сначала их озвучим, а далее я выскажу свое мнение. Итак:
\begin{itemize}
\item     Дейтрейдинг: \$50,000 — \$100,000
\item     Свинг трейдинг: от \$30,000
\item     Позиционный трейдинг и инвестирование: точная сумма не указана, значит будем считать минимальный возможный брокерский счет в \$1,000 — \$2,000.
\end{itemize}

Уверен, что 99\% людей, которые сейчас просматривают эти цифры, считают их нереальными. Когда я только начинал, то у меня также возникла такая мысль (пока сам не увидел, имеется в виду, торговые счета дейтрейдеров в \$100,000). Но, здесь нужно брать во внимание, что Роберт Дил пишет об профессиональных индивидуальных трейдерах, которые зарабатывают себе этим на жизнь. Речь не идет о среднестатистическом трейдере, что наскреб несколько тысяч долларов семейных сбережений (известны случаи, когда одалживают, закладывают имущество и прочее).

Комиссия по ценным бумагам США (SEC) вообще запретила дейтрейдерские счета, которые менее \$25,000 (информация от февраля 2011 года). Но, следует заметить, что берутся во внимание только те трейдеры, которые совершают 4 или более внутридневных сделок за 5 торговых сессий.

Вывод, который можно сделать по этим рекомендациям: чем короче ваш
таймфрейм, тем больший должен быть ваш торговый капитал.

\subsubsection{Мои принципы в определении адекватного брокерского
  счета}

Я сторонник точности и математического расчета. Если бы я только начинал свою торговлю, то следовал бы следующим принципам. Что должен обеспечивать адекватный торговый капитал? Всего две вещи:
\begin{itemize}
\item     Возможность покупать
\item     Возможность сохранять свою покупательную способность на протяжении неопределенного времени.
\end{itemize}

Что значит первый пункт? Минимальная позиция – 1 лот или 100 акций. Я отдаю предпочтение бумагам до \$100. То есть, мой брокерский счет должен позволять мне развернуться на: 100 акций * \$100 = \$10,000.

Так как, брокерские фирмы предлагают кредитное плечо два к одному трейдерам, который оставляют свои позиции овернайт (после закрытия торговой сессии), то свои желанные \$10,000 я могу получить, внеся на счет только \$5,000 (остальные \$5,000 получаю в виде кредита).

Что значит второй пункт? Во-первых, если моя первая сделка будет убыточной, то мой капитал потеряет свою покупательную способность (у меня будет менее \$10,000) и мне придется довольствоваться меньшим. Другими словами, я должен создать запас в несколько тысяч долларов для обеспечения сохранности покупательной способности.

Во-вторых, я не хочу рисковать более 2\% (это как пример) своего капитала, чтобы сохранить его покупательную способность на протяжении неопределенного времени (пока не начну стабильно из месяца в месяц зарабатывать деньги). Например, есть два брокерских счета – на $1,000 первый и $10,000 второй. В любой сделки стоп лосс (ограничитель риска) выставляется не более чем на:
\begin{itemize}
\item     (2\% * \$1,000)/100 = \$20 в первом случае
\item     (2\% * \$10,000)/100 = \$200 во втором.
\end{itemize}

Если вы будете базировать свою стратегию постановки стопов на таком простом методе, как 2ATR, что описан в статье размер позиции – укрощение волатильности, то убедитесь, что средний стоп лосс для 1 лота будет стоит вам \$100. Сравните с полученными результатами и сделайте выводы сами.

Риск в \$100 (еще раз повторюсь, что это в среднем для данной стратегии) – это 2\% от капитала в \$5000.

Мои заключения таковы: если вы знаете, что делаете, то можете открывать счет на \$5000 (это на грани фола). Оптимальным торговым капиталом я считаю не менее \$10,000 (таковым является мой второй счет).

\subsubsection{Нужно ли, вообще, обсуждать эту тему?}

Здесь есть две позиции. Первая – если вы рассматривайте себя, как свинг трейдера или дейтрейдера (то есть, спекулянта), то вопрос о размере торгового капитала, по-моему мнению, должен стоять на втором месте. Главный вопрос – это о стиле трейдинга. Поймите, ваше время важнее денег. Если вы имеете прибыльную работу, у вас маленькие дети или какие-то другие жизненные ситуации, то не нужно лезть из кожи вон, чтобы стать дейтрейдером. Анализируйте или станете в ряды тех, кто говорит, что фондовый рынок это обман и т.п.

И вторая позиция – если вы рассматриваете себя, как инвестора (а это случиться рано или поздно, поверьте), то тема размера торгового капитала для вас пустая трата времени. Как писалось выше можно начинать и с \$1,000. Причем, как пишут в американских финансовых журналах, эра массового ухода людей в дейтрейдинг, начатая в 90-ых годах прошлого века, заканчивается. Люди стали умнее.

Еще хочется писать далее, но пост получился и так слишком затянутым (но, надеюсь полезным). Оставлю немного информации для следующих статей. Все, что хочется сказать в конце – риск менеджмент это основание вашего успеха на фондовом рынке. Изучайте его, продолжая читать блог о трейдинге. Если со стратегией свинг трейдинг вам все понятно, то переходите к странице, посвященной продвинутому свинг трейдингу. Спасибо за внимание и контролируйте свои риски!


Очень познавательно: котировки американских акций

\section{Продвинутый свинг трейдинг}

Продвинутый свинг трейдинг — это обучающий раздел, где мы более
подробно рассмотрим стратегию свинг трейдинга. Здесь вы научитесь
интерпретировать ценовую динамику без применения индикаторов, увидите
на примерах, как открывать позиции в альтернативный способ, узнаете,
как отбирать необходимые акции для торговли откатов при помощи
фильтров за несколько секунд, а также получите несколько хороших
советов. В общем, это достойное дополнение к тому, чему вы научились в
первых двух разделах по основам и стратегиb свинг трейдинг.

\subsection{Альтернативные стратегии входа для агрессивного свинг
  трейдинга}

Здравствуйте, читатели блога о трейдинге. Базовую графическую модель для открытия позиции мы в деталях разобрали в соответствующей статье. Она состоит в основном с трех свечей: точка разворота в основании имеет минимум, еще более низкий минимум и, наконец, более высокий минимум. Для точки разворота на вершине мы ищем максимум, более высокий максимум и третий максимум, который ниже предыдущего.

Эта техника воспринимается проще на графике:

Слева на графике вы видите точку разворота на вершине. Мы будем искать возможность для открытия позиции в день третьей свечи, когда появился более низкий максимум.

Посредине графика вы видите точку разворота в основании. Открывать позицию нужно в конце торговой сессии третий свечи, потому что сформировался более высокий минимум.

Важно понимать, что не все точки разворота будут приводить к смене
тренда. Но все же большинство будет. На графике выше есть еще примеры
базовых графических моделей. Сможете их найти?

\subsubsection{Агрессивные входы}

Агрессивные входы – это открытие позиции, при котором вы покупаете или продаете акцию еще до того, как она сформировала точку разворота. Вы торгуете с ожиданием, что в дальнейшем она образуется.

Смотрите тот же пример выше. Разворотные модели я выделил зеленым цветом. Здесь также присутствуют два последовательных понижающихся минимума, со следующим минимумом, который выше предыдущего (кроме крайнего правого, где третья свеча еще не появилась).

Но посмотрите на вторые свечи этих примеров. Это типичные сильные бычьи дни. Комбинация этих свечей с предыдущими формируют модель свечного анализа – бычье поглощение.

В таких ситуациях агрессивные входы возможны в конце торговых сессий вторых свечей, когда есть уверенность в окончательном формировании моделей бычьего поглощения. При этом мы ожидаем, что на следующий день сформируется более высокий минимум.

Какие еще могут быть примеры агрессивных входов? Это, например такие
свечные модели, как «завеса из темных облаков» и «просвет в облаках»,
«молот» и «повешенный». Как торговать молот мы рассматривали в статье
о свечной модели Т-30. Как формируются данные свечные модели и с чего
состоят, смотрите на странице, посвященной свечным моделям.

\subsubsection{Какая стратегия входа лучше?}

Здесь нет однозначного ответа. Все впирается в соотношение риска к прибыли. Базовая графическая модель менее рискованная, потому что акция уже движется в нужном нам направлении. Но, как иногда бывает, движение цены третьей свечи настолько стремительно, что наша точка входа отдаляется далеко от минимума разворотной модели, где мы, как правило, ставим стоп лосс. Получается, что нам нужно сократить размер позиции, для того, чтобы не поднимать уровень риска.

Альтернативные входы, потому и называются агрессивными, что мы хотим открыть позицию пораньше, еще до появления третьей заключительной свечи. Это позволяет нам поставить более близкий стоп лосс и увеличить размер позиции. Но пока третья свеча не подтвердить нашу правоту, нам остается только ожидать.

Конечно, когда я нахожу акцию с сильной свечной моделью, то отдаю
предпочтение агрессивному входу. Для подтверждения своей правоты можно
воспользоваться меньшим таймфреймом и поискать там какие-то признаки
разворота.

\subsubsection{Почему важно идентифицировать точки разворота?}

Почему важно идентифицировать точки разворота?

Это важно, потому что они указывают нам момент, когда баланс торговых сил перемещается от покупателей до продавцов и наоборот. Например:
\begin{itemize}
\item     Первый день: акция делает максимум
\item     День второй: акция показывает более высокий максимум
\item     Третий день: акция делает еще более высокий максимум
\item     День четвертый: акция поднимается к новому более высокому максимуму
\item     Пятый день: акция показывает более низкий максимум
\end{itemize}

Учтите: это не обязательно должно быть 5 дней.

Что происходит на пятый день? Покупатели могли толкать цену к новым максимумам 4 дня подряд, но на пятый выдохлись. Это значит, что сила быков ослабла, и баланс сил может переключиться на медведей.

Такой же пример вы можете сделать для себя только, когда баланс сил переходит к покупателям. Это просто воспринимается. Каждый день продавцы опускают цену вниз, и вдруг появляется свеча с более высоким минимумом. Кто-то их остановил? Преимущество может перехватить вторая сторона.

Это не должно быть слишком сложно. Просто помните: свинг трейдинг это командная игра. Ваши соперники – другие свинг трейдеры. Каждый старается сыграть за команду победителей. Начнешь игру рано, еще до расстановки сил, проиграешь. Войдешь поздно, когда игра уже сыграна, тоже проиграешь. Базовые и альтернативные графические модели для входа в игру позволят вам это сделать как раз в нужный момент. Блог о трейдинге благодарит за внимание. Будьте успешными!


Следующая статья по трейдингу: когда лучше покупать акции?

\subsection{10 советов по ценовым движениям, которые улучшат вашу
  торговлю}

Приветствую вас, читатели блога о трейдинге. Движение цены акции не хаотично, а закономерно. Выявление этих закономерностей для свинг трейдера – это искусство поиска отдельных свечей, которые предопределяют возможность дальнейшего движения цены, без использования какого-либо индикатора.

В конечном счете, анализ движения цены говорит вам, кто контролирует акцию. Он также указывает момент перехода контроля от покупателей к продавцам или наоборот. Когда вы начнете понимать и интерпретировать движение цены, то сможете достаточно точно определять точки разворота на графиках акций и постоянно зарабатывать деньги.

Изучите и проработайте советы по ценовым движениям на этой странице, и я гарантирую, вы подниметесь на ступень, а то и на две в свинг трейдинге.

Давайте начинать!

\paragraph{Совет №1. Идентификация уровней поддержки и
  сопротивления}

Это нелегкая задача. Идентификация уровней поддержки и сопротивления – это одна из первых вещей, что вы познаете в техническом анализе. Это является наиболее важным аспектом в чтении свечных графиков. Но, как много трейдеров действительно уделяют этому внимание? Не очень много. Большинство увлечено сигналами стохастика, MACD, и другой ерундой.

Некоторые трейдеры думают, что уровни поддержки и сопротивления – это конкретная цена. Ошибочное мнение. Это области на графиках акций. Позвольте привести пример.

Выделенные области являются корректными уровнями поддержки и
сопротивления. Например, мы говорим: «CVX дала откат; поддержка
находится на уровне \$16.75 — \$16.00». Когда трейдер говорит о
какой-то конкретной цене, то он забывает о тенях свечей. Подробнее
читайте в отдельной статье, посвященной уровням поддержки и
сопротивления.

\paragraph{Совет №2. Анализ точек разворота}

Точки разворота – это области на графике акции, где происходят важные краткосрочные смены тенденции. Но не все точки разворота одинаково важны. В действительности, ваше решение о покупке отката должно зависеть от предыдущей точки разворота. Давайте смотреть пример:

Смотрите на разворотную модель в основании отката, указанную зеленной стрелкой. Она находится в области уровня поддержки, и вы можете рассматривать этот откат на покупку.

Но, смотря на предыдущую вершину, возникает две проблемы. Первая, это недостаточно пространства для торговли! Расстояние между откатом и предыдущей вершиной небольшое. Желательно иметь достаточный запас хода для акции, чтобы можно было хотя бы передвинуть стоп в безубыток.

Вторая проблема: предыдущая вершина состоит из группы свечей, которые
формируют сильное сопротивление (синяя область). Цене будет тяжело
продавить его. Для вас лучше торговать откаты, когда предыдущая
вершина формируется одной-двумя свечами.

\paragraph{Совет №3. Поиск широкодиапазонных свечей}

Широкодиапазонные свечи (имеют длинное тело) говорят нам о нескольких вещах. Во-первых, они часто обозначают важные изменения в торговом настроении на любом графике и таймфрейме. Посмотрите на график ниже:

В двух случаях, после значительного нисходящего движения, возникает медвежья свеча с телом больше обычного. Следующая за ней бычья свеча, тоже имеет большое тело и знаменует разворот к восходящему тренду.

Во-вторых, широкодиапазонные свечи указывают направление преобладающего тренда. Смотрите тот же график:

В-третьих, часто откаты разворачиваются в пределах широкодиапазонных
свечей. Это происходит потому, что трейдеры, пропустившие «большое
движение», хотят получить второй шанс на откате. Все тот же график:

\paragraph{Совет №4. Узкодиапазонные свечи ведут к взрывному
  движению}

Узкодиапазонные свечи (с маленькими телами) часто указывают на разворот краткосрочной тенденции. Низкая волатильность, к которой они приводят, может перерасти во взрывное высоковолатильное движение.

Узкодиапазонные свечи говорят нам, что предыдущий импульс
затухает. Покупатели и продавцы пребывают в равновесии, но кто-то из
них должен захватить контроль?

\paragraph{Совет№5. Ищите длинные тени возле ценовых уровней}

На свечном графике, длинные верхние или нижние тени свечей обычно указывают на формирование важных свечных моделей «молота» или «падающей звезды». Неважно, как этот паттерн называется, но если он находится на уровне поддержки, то это говорит об удержании ценового уровня.

Посмотрите на этот молот в основании отката. Это действительно бычья
модель. Продавцы опустили цену далеко ниже уровня, но покупатели
показали характер и закрыли день выше поддержки. Это дало последующий
импульс к росту.

\paragraph{Совет №6. Правило 50\%}

Когда вы видите разворотную модель, как можно сказать, насколько важна заключительная свеча? Просто посмотрите, как она двигается по отношению к предыдущей свече. Если она перекрывает 50\% ее диапазона, значит важна. На графике такие свечи часто формируют модели «поглощение» или «просвет в облаках».

На этом тренде почти каждый откат завершается свечной моделью «поглощение». Это очень сильные модели. Я их люблю торговать.

\paragraph{Совет №7. Гэпы и ценовые модели с ними связанные}

Гэпы на графиках всегда играют важную роль. Но нам наиболее интересно, когда они возникают в основании отката. Тогда формируются модели свечного анализа под названием «звезды». Смотрите пример:

Также на изображении мы видим еще одну полезную функцию гэпов. Гэп,
возникший сразу после разворота, подтверждает движение цены в
выбранном направлении. На графике видно сразу несколько таких
примеров. Более детально читайте «Гэп и его анализ».

\paragraph{Совет №8. Измеряйте глубину отката}

Как далеко цена может откатывать по сравнению с предыдущей волной? Более половины или менее? Ответ на этот вопрос важен, поскольку он может определять будущее движение акции. Даю вам пример:

Здесь на протяжении восходящего тренда есть несколько откатов. Все они опускаются в пределах половины предыдущей волны. И это нормально. Когда цена валится ниже 50\%, нужно задуматься о возможности изменения тенденции.

\paragraph{Совет №9. Последовательные восходящие и нисходящие дни}

Акции изменяют направление движения после последовательных восходящих или нисходящих дней. Помните это, когда ищите откат для открытия позиции. Вот примеры:

Всегда покупайте акции после нескольких медвежьих свечей. Продавайте,
когда откат состоит из последовательных бычьих свечей. Для начинающих
трейдеров такая тактика может звучать нелогично. Ведь нужно покупать
силу, а продавать слабость. В действительности, все наоборот!

\paragraph{Совет №10. Локализация цены на тренде}

Вы, наверное, слышали выражение: «Тренд – ваш друг». Я бы сказал: «Начало тренда – ваш друг». Почему? Потому что большинство мощных движений возникает именно вначале тренда.

На примере акция пробила важный уровень сопротивления. Начался новый тренд. Лучший момент для открытия позиции – это первый откат (указан стрелкой) после пробоя.

Это все. Используйте эту информацию для разработки своих собственных стратегий. Может теперь вам не захочется пользоваться индикаторами вообще. Блог о трейдинге благодарит за внимание. Будьте успешными!


Очень познавательно: что означает фьючерс?

\subsection{Графическая модель Т-30 и как ее торговать}

Приветствую вас, читатели блога о трейдинге. Данная графическая модель
– это моя «каждодневная» модель. Она является достаточно надежной, и
поэтому я ее торгую наиболее часто. Ее просто определить на графике и
легко торговать.

\paragraph{Установка}

Модель носит название Т-30 из-за своей длинной тени, которая пересекает 30-периодное скользящее среднее (30 EMA). Ее задача – выбросить других трейдеров с акции. Т-30 вам может напоминать «молот» из свечных моделей. И это верно. Далее мы ее так и будем называть. Но она может и не отвечать всем критериям идеального молота. Главное длинная тень. Цвет тела не имеет значения.

Учтите: особенности в 30-периодном скользящем среднем нету. Это просто рекомендация. Ищите по графику влево от паттерна уровни поддержки и сопротивления.

Специфика торговли любой графической модели с длинной тенью состоит в том, что ее объем торгов должен быть выше предыдущих свечей. Это подтверждает, что многие трейдеры были выброшены, и сформировалась поддержка.

Вот пример:

\paragraph{Как торговать эту графическую модель}

Есть несколько вариантов работы с Т-30, что зависит от вашего ожидания
риска/прибыли. Я покажу вам несколько примеров.

\subsubsection{Открытие позиции}

Если у вас есть возможность торговать на протяжении дня, тогда покупайте акцию в день образования молота (Т-30) ближе к закрытию. Не нужно ждать никаких «подтверждений» или еще чего-то. Единственное, что вам необходимо, это уровень поддержки на графике акции и возросшие объемы торгов.

Если у вас нет возможности торговать на протяжении дня, то расположите покупочный стоп ордер (stop buy) выше максимума дня образования молота. На следующий день проверьте, исполнился ли ваш ордер. Если да, то ставьте стоп лосс для ограничения убытков, если цена не пойдет в предполагаемом направлении.

\subsubsection{Стоп лосс ордер}

Существует два варианта для постановки вашего стоп лосс ордера. Оба
имеют преимущества и недостатки. Вы решаете, что лучше для вас.

\paragraph{Вариант 1:}

Выставьте стоп лосс сразу под минимум молота. Преимущество такого расположения: ваш стоп будет достаточно далек от точки входа и обезопасит от преждевременного вылета из позиции.

Недостатком здесь будет меньший размер позиции.

\paragraph{Вариант 2:}

Перейдите на 60-минутный таймфрейм и расположите свой стоп под областью поддержки максимально близко к телу свечи. Преимущество такого стопа: вы можете позволить себе купить больше акций.

Конечно, ваша позиция может преждевременно быть закрыта, поскольку защитный ордер будет стоять слишком близко от точки входа. Это является недостатком.

Лично я предпочитаю первый вариант. Для меня не проблема, если куплю
немножко меньше акций. Мне нравится, когда мой стоп лосс пребывает вне
«рыночного шума». Тогда я могу сесть и спокойно ждать, когда цена
двинется в мою сторону.

\paragraph{Фиксация прибыли}

Когда вы торгуете широкодиапазонные свечи, как молот, то можете заметить, что иногда акция торгуется в коридоре следующие день-два. Это нормально. Вы уже открыли позицию по акции и просто ждете, когда войдут остальные трейдеры. Кроме того, свечи, следующие за молотом, обычно низковолатильные, с узким диапазоном, как например звезды или доджи.

Будьте терпеливы! Не позволяйте волнению двигать ваш стоп
выше. Дождитесь, пока акция действительно начнет движение в
предполагаемом направлении, и тогда переставьте его, используя вашу
стратегию закрытия позиции.

\subsubsection{Торговые советы}
\begin{itemize}
\item     Выбирайте те акции, в которых тело молота находится возле 30-периодной скользящей средней. Вы же хотите, чтобы побольше трейдеров были выброшены с этой акции, перед вашим входом.
\item     Эти же установки работают для коротких продаж. Только теперь вам нужно искать бумаги со свечной моделью типа «падающая звезда», тень которой пересекает 30 EMA.
\item     У позиции будет бOльшее преимущество, если имеются дополнительные тени в предыдущие дни.
\item     У позиции будет бOльшее преимущество, если между текущим молотом и предшествующей ему свечой имеется гэп.
\item     Всегда ищите на графике уровни поддержки и сопротивления.
\end{itemize}

\subsubsection{Когда хороший паттерн становится плохим}

Да, вы будете иметь убытки, торгуя эту графическую модель. Нет такого паттерна, который бы гарантировал все прибыльные сделки! Но, правильно управляя сделкой, деньгами и своими эмоциями, вы сможете хорошо зарабатывать с этой моделью. Далее мы поговорим об еще одной хорошей графической модели. Блог о трейдинге благодарит за внимание. Будьте успешными!


Стоит посмотреть - свечной анализ: книги

\subsection{Как торговать низковолатильные развороты?}

День добрый читатели блога о трейдинге. Что происходит, когда трейдеры
игнорируют акцию? Рисуются свечи с узким диапазоном и низкой
волатильностью. Такую картину на графике можно ассоциировать с
городом-призраком – пустыня. Будьте начеку, так как за этим затишьем
может последовать взрывное движение!

\subsubsection{Установка}

Вы видите откат в активную зону на объемах, который заканчивается серией узкодиапазонных свечей. Это, как правило, доджи или звезды, которые сопровождаются низкими объемами торгов. Скомбинировав эти два факторы: свечи с узким диапазоном и низкие объемы торгов, – вы получаете выигрышную комбинацию в действие.

Почему нам следует обращать внимание на такие низковолатильные места? Потому что мы уже знаем, что за низкой волатильностью следует высокая. И наоборот. Это повторяется постоянно, на всех таймфреймах.

Учтите: в свечной терминологии, «звезда» должна иметь ценовой разрыв (гэп) с предыдущей свечой. При торговле низковолатильных разворотов это не обязательное условие.

Посмотрите на пример:

Все, на что нужно обратить внимание, выделено. Состоялся откат в активную зону со следующей за ним серией из 6 свечей с узкими диапазонами (доджи и звезды, кроме одной). Смотрите, как упали объемы торгов. Нисходящий импульс замедлился, и акция ждет следующего толчка.

Это то место, где входят профессиональные трейдеры.

Еще один пример:

\subsubsection{Как торговать эту графическую модель}

Здесь могут возникать сложности. Если мы будем входить привычным способом, когда текущая свеча закрывается выше максимума предыдущей, то из-за возможности взрывного движения (как во втором примере) цена может уйти слишком далеко от графической модели. Хотелось бы войти пораньше с маленьким стопом и риском.

\paragraph{Открытие позиции}

Как уже говорилось, наша задача получить сделку до начала взрывного движения. Вот несколько вариантов.

\subparagraph{Вариант 1:} выставьте покупочный стоп ордер (buy stop) над самым высоким максимумом из серии узкодиапазонных свечей. В наших случаях, это третья и вторая свечи соответственно.

\subparagraph{Вариант 2:} дождитесь появления другой графической модели. В первом примере я выделил красным знакомую нам свечную модель. Помните Т-30?

\subparagraph{Вариант3:} перейдите на меньший таймфрейм, скажем часовой график, и дождитесь его пробоя. Часто такой подход дает вам ранние сигналы о возможности разворота акции в предполагаемом направлении.

Конечно, самое тяжелое в торговле этой моделью – это открытие позиции. Никогда не знаешь, сколько будет продолжаться низковолатильный разворот, как себя цена поведет после него. Если движение будет плавное, то можно входить в акцию при помощи нашей стандартной графической модели.

За что я ценю этот паттерн, так это за высокое отношение прибыли к
риску. Так как волатильность низкая, соответственно небольшой риск. Но
последующее движение, обычно, значительное.

\paragraph{Фиксация прибыли}

Здесь нет ничего особенного. Просто передвигайте стоп лосс, используя любимую стратегию закрытия позиции. Хотя, если рынок предлагает вам подарок, не отказывайтесь! Например, если цена всего за пару дней значительно выросла, скажем, на 15\%, зафиксируйте часть прибыли, а оставшейся позиции позвольте расти дальше.

\subsubsection{Торговые советы}
\begin{itemize}
\item     Низкая волатильность ведет к высокой волатильности
\item     Высокая волатильность ведет к низкой
\item     Свечи с узким диапазоном предупреждают о замедлении текущего импульса. Ждите разворота!
\item     Всегда ищите уровни поддержки и сопротивления на графике.
\end{itemize}

Самые прибыльные сделки открываются, когда другие их не замечают. Торгуйте низковолатильные развороты, пока остальные пребывают в нерешительности. Это принесет вам щедрые плоды. Блог о трейдинге благодарит за внимание. Удачных торгов!

\subsection{Графическая модель «ложный разворот» и как ее торговать}

Здравствуйте, читатели блога о трейдинге. Вы будете наблюдать эту
графическую модель всегда. Ее пример можно найти за 5 минут,
просматривая графики акций. Я расскажу вам, как с ней работать и как
ее понимать. Это одна из наиболее надежных графических моделей,
которую я знаю. Вы поймете почему, уже через минуту.

\subsubsection{Установка}

Как ясно из названия, эта графическая модель является «ловушкой» для свинг трейдеров (и моментум трейдеров) как раз в средине движения.

Будем разбирать на примере:

Я выделил зону из 7 свечей, которая нас интересует. Произошел отличный откат в активную зону. Трейдеры ждут, когда же произойдет разворот, чтобы купить эту бумагу на закрытии выше максимума предыдущей свечи. Такой момент возникает на 4-ой свече выделенной зоны. Стоп лосс ставиться ниже минимума волны (1-я свеча).

Далее 5-я бычья свеча с большим телом подтверждает правильность торгового решения. Акция поднимается где-то к средине отката. Все хорошо!

Вдруг, следующие два дня цена уходит ниже минимума волны и бьет все стопы. Что произошло! Все недоумевают. Друзья, это ложный разворот в действии.

Учтите: последняя свеча этого паттерна не должна закрываться возле своего минимума или за уровнем поддержки. Очень часто формируется молот (или графическая модель Т-30), длинная тень которого сбивает стопы и закрывается вверх.

В волновой теории Эллиота эта графическая модель известна, как А-В-С волны, только в меньшем масштабе. Мы называем ее как «ложный разворот», и думаю, что вы поняли почему.

Рассмотрите еще один пример без моих комментариев:

\subsubsection{Как торговать эту графическую модель}

Ключевым моментом здесь является «встряска» свинг и моментум
трейдеров. Финальная волна должна уйти ниже минимума первой. Повторюсь
снова, часто эта финальная волна заканчивается молотом (лучше такие
графики и ищите). Этот молот снесет все стопы и подготовит вам позицию
для входа!

\paragraph{Открытие позиции}

Ничего нового я здесь вам не скажу. В наших обоих случаях есть сформированные молоты. Мы открываем позицию в день их появления под закрытие сессии.

Если никакой свечной модели вы не наблюдаете, то входите нашим стандартным способом открытия позиции, когда формируется точка разворота в основании.

Как вы будете открывать позицию – дело за вами.

\paragraph{Стоп лосс ордер}

Тоже, ничего особенного. Просто поставьте защитный стоп там, где в нем
наибольший смысл. Как правило, это под минимумом волны (разворотной
свечи). Не забывайте также искать уровни поддержки и сопротивления.

\paragraph{Фиксация прибыли}

Здесь работаем по стратегии закрытия позиции. Если вы открыли позицию
вначале второй фазы рынка и ожидаете серьезного движения, то
передвигать стоп лосс нужно с запасом хода для цены акции. В таких
случаях я это делаю на недельных свечах. Тогда такую сделку
рассматриваем, как трендовую, а не волновую.

\subsubsection{Торговые советы}

\begin{itemize}
\item     Смысл этой графической модели в том, чтобы финальная волна опустила цену ниже минимума первой. Это решающий фактор.
\item     Вы будете находить такую же графическую модель и на нисходящем рынке для коротких продаж. Она будет перевернута.
\item     Ложный разворот появляется на всех таймфреймах, не только на дневном.
\end{itemize}

После небольшой практики, вам понравится торговать эту модель. Она представляет краткосрочный экстремум на рынке, где множество продавцов «вылетели» с акции перед вашим входом. Блог о трейдинге благодарит за внимание. Будьте успешными!


Интересно: какие акции приносят большие дивиденды?

\subsection{Графическая модель «ложный пробой консолидации» и как ее
  торговать}

Добрый день, читатели блога о трейдинге. Эта графическая модель
является хорошим примером того, почему большинство трейдеров теряют
деньги, торгуя акциями. Они попадаются в ловушку ложных маневров
рынка, после которых, как правило, возникают мощные движения в цене
акции.

\subsubsection{Установка}

Мы уже с вами рассмотрели такую графическую модель, как ложный разворот. Паттерн, который мы разберем в этой статье, подобный предыдущему тем, что здесь трейдеры также «вытряхиваются» с акции ложными движениями ее цены.

Давайте начнем. Смотрите график ниже:

Я выделил две области после отката акции в активную зону. Что происходит в желтом прямоугольнике? На протяжении шести дней цена движется в боковом диапазоне. Так как тренд восходящий, вы, наверное, ожидаете продолжения движения вверх. И это правильно. Некоторые трейдеры покупают акцию во время такой консолидации, чтобы уменьшить риск. Но в действительности, пока не произошел пробой, консолидацию торговать нечего.

Учтите: тысячи трейдеров видят, как и вы, это боковое движение. И они думают так же, как и вы.

Теперь перейдем к области, выделенной красным. Цена прорвалась сквозь консолидацию ниже! Причем сформировалась сильная медвежья свеча с большим телом и закрытием возле минимума. Те трейдеры, которые покупали в желтой области со стопом ниже уровня консолидации, скорее всего, вылетели из позиций.

После такой нисходящей свечи, эта бумага выглядит не такой уж привлекательной для покупки.

Но следующая свеча в красной области является прямой противоположностью ее предшественнице. О чем это может говорить?

Учтите: когда покупатели и продавцы запирают цену в узких границах,
акции приходится делать ложные маневры, чтобы «раструсить» позиции и
продолжить движение. Это тот случай, когда нужно сделать шаг назад,
чтобы продвинутся на два вперед.


\subsubsection{Как торговать эту графическую модель?}

Здесь нужно выделить три момента:
\begin{itemize}
\item     Консолидация
\item     Пробой
\item     Разворот
\end{itemize}

Вам необходима акция с боковым движением, ложным пробоем вниз ниже уровня консолидации и, наконец, разворотной моделью в основании. Потому-то эта графическая модель и называется ложный пробой консолидации. В нее попадают те трейдеры, которые торгуют консолидацию, ожидая продолжения тренда.

Просмотрите еще несколько примеров:

\paragraph{Открытие позиции}

Входить нужно во время разворотной свечи. Но, здесь вам не подойдет стандартная модель разворота. Разворотная свеча должна быть мощной. Убедитесь в том, что она закрывается хотя бы на уровне половины диапазона предыдущей, пробойной свечи.

Если говорить терминами свечного анализа, то у вас должна
сформироваться, либо свечная модель «просвет в облаках», либо
«поглощение».

\paragraph{Фиксация прибыли}

Здесь есть один нюанс. Графическая модель «ложный пробой консолидации» еще известен, как «рогатка». Почему? После ложного движения, которое выступает рукоятью рогатки, консолидация может продолжаться еще некоторое время. Поэтому начальный защитный ордер, который вы выставляете после открытия позиции под минимум разворотной модели, должен сохраняться до выхода цены из консолидации в предполагаемом направлении.

Просмотрите примеры, которые приведены выше еще раз, и вы увидите модели «рогатки» практически на всех, кроме одного (второго).

Когда акция пробьет зону консолидации в предполагаемом направлении,
перемещайте свой стоп лосс, как мы это делаем при стратегии выхода
свинг трейдера.

\subsubsection{Торговые советы}

\begin{itemize}
\item     Чем длительнее консолидация, тем больший потенциал последующего движения.
\item     Разворотная свеча должна быть сильной: большое тело, с поглощением, либо перекрытием большей части диапазона предыдущей, пробойной свечи.
\item     Объемы торгов значения не имеют, но вы можете отметить их увеличение во время ложного пробоя.
\item     Если весь рынок (S\&P 500) проторговался в этот день сильно вниз (большая медвежья свеча), то потенциал этой графической модели увеличивается.
\end{itemize}

Есть две вещи, которые мне нравятся в этой модели. Во-первых, позиция открывается после волны продаж. Во-вторых, торговля ведется против рыночной толпы. А это компоненты успешного свинг трейдинга. Блог о трейдинге благодарит за внимание. Удачных торгов!


Думаете, что бы еще полезненького почитать? - правила стратегии риск менеджмента

\subsection{Чем могут быть полезны уровни Фибоначчи свинг трейдеру?}

Приветствую вас, читатели блога о трейдинге. Уровни Фибоначчи могут быть полезны свинг трейдеру для идентификации разворотов на графиках акций. На этой странице мы поговорим о последовательности Фибоначчи и рассмотрим на примерах, как можно торговать, применяя уровни Фибоначчи.

\paragraph{Последовательность Фибоначчи}

Числа Фибоначчи были разработаны Леонардо Фибоначчи и представляют собой последовательность цифр, в которой каждая следующая цифра является сумой двух предыдущих. Вот пример:

1, 2, 3, 5, 8, 13, 21, 34, 55 …

Видите, когда вы суммируете 1 и 2, то получаете следующую цифру в последовательности – 3. Далее додаете 2 к 3 и получаете 5. И так до бесконечности.

Хорошо! Но, как эта последовательность может помочь свинг трейдеру?
Подмечено, что отношения между этими числами дают нам уровни
Фибоначчи, которые широко используются в техническом анализе.

\subsubsection{Уровни Фибоначчи, которые чаще всего используются}

Акции часто перед тем, как развернутся, откатывают или восстанавливаются на определенный процент от предыдущего движения. Эти проценты и определяют уровни Фибоначчи: 38.2%, 50%, 61.8%. В действительности, значение в 50% ничего общего с уровнями Фибоначчи не имеет. Просто тренд настолько часто разворачивается, когда проходит полпути предыдущего движения, что трейдеры включили его в эту графическую модель.

Ниже смотрите графическое объяснение сказанного:

На этом рисунке изображено графическое представление точек разворота цены акции на восходящем тренде. Для нисходящего тренда аналогично, только с точностью наоборот.

После того, как бумага преодолела некоторый путь вверх (А), она возвращает часть этого движения (В) для того, чтобы с новыми силами снова двинутся в ожидаемом направлении ©. Эти возвращения или откаты вы и хотите видеть на графике, когда ищите возможность для открытия позиции.

Как только акция начала свое возвратное движение (откат), вы рисуете уровни Фибоначчи на ее графике, чтобы определить возможный момент разворота. Только не нужно покупать бумагу на автомате, потому что она достигла какого-то значения Фибо! Ожидайте и следите за развитием разворотной свечной модели возле области 38.2\%. Если вы ее не видите, то цена может опуститься до уровня 50\%. Ищите признаки разворота здесь.

Вы не можете знать, как и когда бумага будет разворачиваться возле уровней Фибоначчи! Просто отмечайте эти области на графике и ожидайте сигнала для покупки или продажи.

\paragraph{Как наносить уровни Фибоначчи на график}

Итак, как пользоваться уровнями Фибоначчи на практике. Просто рисуем сетку Фибо, используя нижнюю и верхнюю точку волны. Пример:

Наверное, каждое программное обеспечение для трейдинга имеет в своем арсенале сетку Фибо. Я использую Thinkorswim, в котором подобных инструментов предостаточно даже для самых требовательных пользователей.

И все же, если у вас нет возможности пользоваться разным торговым софтом, вспоминайте уроки математики и наносите уровни Фибоначчи на график вручную.

Рассчитываете диапазон движения волны путям вычитания цены ее минимума из максимума. Далее умножаете полученное значение на нужные проценты: 38.2\% (0.382), 50\% (0.5), 61.8\% (0.618). Наконец, отнимаем последнее число от цены максимума волны. Если кратко:

Мах – (Мах — Мin)*0.382 или *0.5 или *0.618

На графике выше показана ситуация, которую я торговал. MCD откатила в активную зону и сформировала перевернутый молот как раз на уровне 50\%. Это дало мне сигнал открывать длинную позицию.

\paragraph{Есть ли в этом польза?}

Хороший вопрос! Возможно… иногда…

В большинстве случаев, когда вы будете накладывать уровни Фибоначчи на график, увидите, что они совпадают с областями поддержки и сопротивления, которые и так хорошо видны. Так что, в действительности, нет нужды загромождать график дополнительными линиями. Вместо этого, вы можете просто, посмотрев на акцию определить, где находятся уровни.

Посмотрите еще раз на график MCD выше. Чтобы определить, где акция, скорее всего, развернется, не нужно никаких уровней, поскольку там присутствует сильная область поддержки вокруг \$99.

Хотя, всегда бывают случаи, когда поблизости нет никаких уровней. Это хорошая предпосылка для использования уровней Фибоначчи. Что же, думаю вреда от использования дополнительных инструментов нет. А вдруг уровни Фибоначчи сделают вашу торговлю проще и прибыльнее. Выбор всегда за вами. Блог о трейдинг благодарит за внимание. Будьте успешными!


Это познавательно: опцион что это значит

\subsection{Знаете ли вы, что такое расширения Фибоначчи, и как их
  использовать?}

Доброго времени суток, читатели блога о трейдинге. Расширения
Фибоначчи являются менее известными, чем уровни Фибоначчи. Часто ли вы
слышите, как кто-то говорит о них. По крайней мере, я нет. Если вы
также не знаете, что такое расширения Фибоначчи, то возьмите карандаш
и лист бумаги, будем работать.

\subsubsection{Что такое расширения Фибоначчи?}

Мы уже знаем, что уровни Фибоначчи измеряют продолжительность отката, то есть движения акции, которое отыгрывает часть предыдущей волны. И, обычно, эта часть составляет или 38.2\%, или 50\%, или 61.8\%.

Но, после разворота, тренд продолжается. Волна, следующая за откатом, должна превышать его более чем на 100\% (чтобы тренд продолжался). И здесь возникает вопрос: можно ли измерить продолжительность этой волны? Да. Измеряется она относительно продолжительности отката при помощи расширения Фибоначчи.

Смотрите пример на графике:

Первый пример левее. Бумага AIR с уровня \$25.92 откатила к \$23.11. Следующая волна превысила отметку \$25.92. То есть, акция выросла более, чем на 100\%, от продолжительности отката. Последняя волна зафиксировалась на \$26.68.

На графике мы не видим уровня сопротивления, в который могла бы упереться цена. Но, можно ли было предполагать, что акция остановится в районе \$26.68? Да. Для этого нужно рассчитать расширения Фибоначчи.
\begin{itemize}
\item     Измеряем диапазон отката: \$25.92 — \$23.11 = \$2.81
\item     Величину диапазона умножаем на коэффициент 1.272: \$2.81 * 1.272 = \$3.57
\item     И финальное действие: \$23.11 + \$3.57 = \$26.68
\item     \$26.68 – это и есть вершина последней волны.
\end{itemize}

Во втором примере (правее):

\$25.0 + (\$28.75 — \$25.0)*1.272 = \$29.77

Вершина последней волны находится на \$29.80. Так что наши расчеты не дотянули 3 цента до равности.

Учтите: есть два наиболее использованных коэффициента для определения расширения Фибоначчи – 1.272 и 1.618. Есть и другие, но эти два работают лучше всех.

Вернемся к графику выше. Вы заметили, что я нанес дополнительные линии. Это и есть расширения Фибоначчи, которые появляются автоматически, как только я измерю продолжительность отката. Все это возможности торговой платформы Thinkorswim.

\subsubsection{Практическая польза}

Расширения Фибоначчи являются полезным инструментов, когда акция торгуется на новом максимуме или минимуме. То есть, когда на графике поблизости мы не видим уровня сопротивления, в который могла бы упереться цена.

Когда у вас открыта длинная позиция, и акция поднимается к новому максимуму, вы наносите на график расширения Фибоначчи и получаете представление, где цена может остановиться. На этих уровнях можно зафиксировать часть прибыли, как это описано в посте «Риск менеджмент: стратегия для начинающих».

Аналогично действуйте, когда имеете короткую позицию. Но, никогда не основывайте свои решения о покупке или продаже только полагаясь на расширения Фибоначчи. Ждите появления свечных моделей (движения цены), которые подтверждают возможность изменения направления тенденции возле целевых областей. Блог о трейдинге благодарит за внимание. Не забывайте, цена – королева на графике!


Полезная информация: характеристика и классификация облигации

\subsection{Стратегия множественных таймфреймов: как использовать в
  свинг трейдинге?}

Здравствуйте, читатели блога о трейдинге. Просмотр разных таймфреймов может дать вам лучшее представление о том, что творится с акцией. Выбором для свинг трейдинга могут стать следующие периоды: дневной, недельный, часовой, 5-минутный.

Для начинающего трейдера анализ одной акции на разных таймфреймах может быть сложной работой, а зачастую и запутывающей. Почему? Потому что разные таймфреймы даже одной и той же бумаги выглядят по-разному! Акция может хорошо смотреться на дневке, но ужасно на 5-минутке.

Я попробую на этой странице как можно наиболее доступно рассказать о
стратегии использования множественных таймфреймов. Но если вы
чувствуете, что в голове появляется каламбур, то
остановитесь. Оставайтесь при дневном и часовом графике. А для
большего развития почитайте на досуге книгу Robert Miner “High
probability trading strategies”.

\subsubsection{Дневной таймфрейм}

Это основное ваше поле деятельности, как свинг трейдера. Когда вы фильтруете и просматриваете акции, то делаете это именно на дневке. Здесь мы ищем торговые возможности.

Дневной график на вашем мониторе должен покрывать период от 5 до 7 месяцев или даже дольше. Это для того, чтобы вы могли видеть и находить уровни поддержки и сопротивления. Помните, что одна свеча отображает одну торговую сессию.

Посмотрите на пример:

Конечно, ваш график будет намного больше этого. Растяните его на весь монитор. Чем больше, тем лучше.

Теперь о содержимом на график. Акция дала откат в активную
зону. Текущая свеча поглотила предыдущую, которая является
перевернутым молотом. Установка отличная! Было бы неплохо узнать об
уровнях поддержки и сопротивления. Здесь лучше всего открыть недельный
график.

\subsubsection{Недельный таймфрейм}

Недельный график позволяет вам сделать шаг назад и посмотреть картину в целом. Дневка более фокусирована на текущей ситуации, поэтому на ней трудно увидеть, что в действительности творится с акцией.

Главная задача недельного графика – это показать вам основное направление тренда и наличие важных технических моделей. Многие инвесторы и институциональные трейдеры основывают свое решение о покупке или продаже бумаги именно на этом таймфрейме. Так что спросите себя: «Если бы я был ими, то купил бы сейчас эту акцию?».

Давайте смотреть на наш пример с этой позиции:

Здесь вам нужно увидеть период от 2 до 3 лет (каждая свеча отображает 1 неделю торгов). Нет оснований углубляться дальше для свинг трейдинга. И вам не следует слишком долго задерживаться на этом таймфрейме.

Теперь о ситуации нашей акции на недельном графике. Тренд
восходящий. Последние 6 месяцев формировалась техническая модель
«голова-плечи». Уровень в области $70-$71, который может стать
поддержкой, если цена закрепится.

\subsubsection{Часовой таймфрейм}

На часовом графике каждую свечу дневки можно рассмотреть, как муравейник под увеличительным стеклом. Откат на дневке, который имеет последовательность из понижающихся минимумов (для покупки), на часовом таймфрейме выглядит, как нисходящий тренд.

Вам нужно искать пробой линии этого нисходящего тренда.

Когда мы говорим, что покупаем акцию выше предыдущего максимума на дневке, то это значит, что покупаем пробой линии тренда на часовом графике.

Давайте смотреть наш пример выше на этом таймфрейме:

Часовой таймфрейм должен охватывать 15-20 торговых сессий (каждая свеча представляет один час).  Вы можете подметить, что анализируя его, проще увидеть (и возможно, предугадать) прорыв линии тренда.

В примере выше цена одним мощным движением прорвала нисходящую линию
тренда. Причем, вы уже заметили, сформировалась графическая модель
«двойное дно». Настало время подобрать момент для открытия позиции.

\subsubsection{5-минутный таймфрейм}

Здесь мы будем искать точку для входа. На дневке у нас присутствует хорошая установка: модель свечного поглощения. Значит, будем покупать во время формирования модели, в конце текущей торговой сессии.

Смотрим наш пример далее:

5-минутный график должен содержать 1-2 дня (свеча отвечает 5 минутам). Вас интересует исключительно текущий день. В нашем случае сформировался восходящий треугольник с уровнем около \$72. Там можно и покупать.

Заменой 5-минутного таймфрейма вам может служить 15-минутный. Дело в том, что дейтрейдеры иногда делают 5-минутный график слишком резким. Используя 15-минутку, можно сделать график более гладким и плавным.

Вы даже можете решить использовать 15-минутный таймфрейм вместо
5-минутного.

\subsubsection{Торговые советы}

\begin{itemize}
\item     Один таймфрейм влияет на другой. Новости отображаются сначала на 5-минутке. Далее это влияние передается на часовой график, дневной и наконец, недельный.
\item     Вы можете найти лучшую позицию для постановки близкого стопа на часовом таймфрейме.
\item     Используйте 200-периодное скользящее среднее на всех таймфреймах – даже внутри дня. Вы будете удивлены, как реагирует акция, когда цена приближается к этой линии.
\end{itemize}

Я вам привел пример на этой странице того, что можно включить в свою стратегию. Чем больше сигналов подтверждения своей правоты вы получите, тем больше у вас шансов открыть прибыльную позицию. Но, в то же время, уменьшится количество трейдов.

Если вас заинтересовала эта тема, то почитайте книгу Robert Miner “High probability trading strategies”. Там вы найдете не только информацию о стратегии множественных таймфреймов, но и другие полезные вещи. Я взял на вооружение несколько советов с этой книги. Блог о трейдинге благодарит за внимание. Будьте успешными!


Полезно знать - перспективы развития российского рынка акций

\subsection{Основы продажи без покрытия}

День добрый, читатели блога о трейдинге. Продажа без покрытия (от англ. short) производится, когда предполагается понижение акции в цене. Фондовая биржа – это одно из немногих мест, где можно продать то, чем не владеешь (что не было куплено). Если вы продаете те акции, которых у вас нет, то вы продаете без покрытия. Для того, чтобы это сделать, необходимо одолжить акции в брокера.

Интересно? Читайте дальше…

\subsubsection{Что такое продажа акций без покрытия?}

Продать без покрытия – это открыть короткую позицию у брокера, одолжив некоторое количество акций. Когда цена понизилась, то одолженные акции выкупаются (покрытие короткой позиции), а разница в котировках является прибылью. Давайте разбирать на примере:

Microsoft торгуется по цене \$30 за акцию. Вы предполагаете, что цена должна понизиться, так что продаете без покрытия 200 акций (\$6000). Ваш брокер готов одолжить вам это количество бумаг Microsoft по цене \$30 за штуку. Microsoft через некоторое время упала до \$20 за акцию.

Вы принимаете решение обналичить свою короткую позицию и покупаете (называется покрытие) 200 акций по \$20 за штуку. Ваш брокер теперь возвращает одолженные бумаги их владельцам, а вы получаете прибыль от разницы в котировках: \$30-\$20=\$10 на 200 штук – это \$2000.

Если вам сложно понять, как это — зарабатывать на том, чем вы не владеете, то думайте в таком направлении:

В объявлении вы увидели, что человек срочно хочет купить планшет за
\$1000 c определенными параметрами. Вы знаете, в каком интернет
магазине  можно купить точно такую вещь за \$800 и предлагаете свои
услуги. Получили \$1000, заказали человеку планшет за \$800, а разницу в
\$200 положили себе в карман. Вы продали то, чем даже не владели.

\paragraph{Этично ли продавать без покрытия?}

Некоторые люди говорят, что продавать без покрытия это неэтично, поскольку поддерживается дальнейшее снижение цены акции. Это ложь! Помните, что после того, как вы продали некоторое количество бумаг, нужно их выкупить назад! Это создает покупательное давление на акцию.

Продавцы, когда покрывают свои короткие позиции по ходу снижения цены,
наоборот поддерживают акцию. Если бы их не было, то рынок мог бы
стремительно обвалиться!

\paragraph{Что такое короткое сжатие?}

Трейдеры, которые практикуют продажи без покрытия, подвержены
дополнительному риску – короткому сжатию. Когда существует повышенный
короткий интерес, может возникнуть дефицит предложения. То есть,
трейдер готов покрыть свою короткую позицию, но акций по такой низкой
цене нет. Тогда покрытие ведется по растущей цене, что уменьшает
прибыль.

\paragraph{Стоит ли мне продавать без покрытия?}

В продажах без покрытия нет ничего плохого. Это просто часть каждодневной работы спекулянта. Покупка акций представляет только пол-уравнения! Если рынок в целом движется вниз, то вам, наверное, не будет хотеться покупать. Так что, чтобы постоянно зарабатывать на фондовой бирже, вам необходимо учиться искусству продаж без покрытия.

Также вам легче будет понимать, где и когда происходят развороты тенденции. Потренировавшись в продажах акций без покрытия, вы начнете понимать, когда другие трейдеры продают, а когда покрывают свои короткие позиции. А это и определяет момент разворота.

В большинстве своем, короткие позиции приносят денег больше и быстрее. Почему? Что легче: толкать вверх, или спускать вниз? Страх является более сильной эмоцией, чем жадность. Цена падает быстрее, чем растет!

Многие участники рынка, как например инвесторы, торгуют только в одном направлении: они покупают. Они не понимают, как можно делать деньги на продажах. Их мышление: «Если цена движется вверх – хорошо, а когда вниз – плохо». Все точка!

Это ошибка. Рынок всегда хорош. Только нужно понимать, когда ты должен покупать, а когда продавать.

Когда я слышу по ТВ, как ужасен медвежий рынок, сколько горя он
приносит людям, то мне, кажется, что Уолл-Стрит просто не хочет, чтобы
все участники рынка знали о продажах акций без покрытия.

\paragraph{Как продавать без покрытия?}

Как свинг трейдер, вы ищите акцию, которая восстановилась (дала отката) в активную зону трейдера. Посмотрите на пример графика:

На графике вы видите, что 10-периодное скользящее среднее находится ниже 30-периодного. Также, цена торгуется ниже 200-периодной простой скользящей средней.

Акция поднялась в активную зону и сформировала точку развороту на вершине, которая отвечает нашим стандартным правилам открытия позиции. Это как раз то место и момент, когда вам нужно продавать без покрытия. Торгуйте в противоположность тому, что вы делаете, когда открываете длинную позицию.

Это все! Когда вы сможете уверено торговать как на бычьем, так и на медвежьем рынках, то будете чувствовать себя более комфортно и прибыльно. Деньги можно делать независимо от направления рынка. В следующем посте мы обсудим, как искать и торговать акции, которые пребывают в сильном тренде. Блог о трейдинге благодарит за внимание по теме продажа без покрытия. Удачных торгов!


Интересно: все комбинации японских свечей

\subsection{Как в свинг трейдинге можно применить индикатор ADX?}





\end{document}